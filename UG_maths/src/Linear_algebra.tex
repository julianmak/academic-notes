\documentclass[letter-paper]{tufte-book}

%%
% Book metadata
\title{Linear Algebra 2H}
\author[]{B. S. H. Mithrandir}
%\publisher{Research Institute of Valinor}

%%
% If they're installed, use Bergamo and Chantilly from www.fontsite.com.
% They're clones of Bembo and Gill Sans, respectively.
\IfFileExists{bergamo.sty}{\usepackage[osf]{bergamo}}{}% Bembo
\IfFileExists{chantill.sty}{\usepackage{chantill}}{}% Gill Sans

%\usepackage{microtype}
\usepackage{amssymb}
\usepackage{amsmath}
%%
% For nicely typeset tabular material
\usepackage{booktabs}

%% overunder braces
\usepackage{oubraces}

%% 
\usepackage{xcolor}
\usepackage{tcolorbox}

\newtcolorbox[auto counter,number within=section]{derivbox}[2][]{colback=TealBlue!5!white,colframe=TealBlue,title=Box \thetcbcounter:\ #2,#1}                                                          

\makeatletter
\@openrightfalse
\makeatother

%%
% For graphics / images
\usepackage{graphicx}
\setkeys{Gin}{width=\linewidth,totalheight=\textheight,keepaspectratio}
\graphicspath{{figs/}}

% The fancyvrb package lets us customize the formatting of verbatim
% environments.  We use a slightly smaller font.
\usepackage{fancyvrb}
\fvset{fontsize=\normalsize}

\usepackage[plain]{fancyref}
\newcommand*{\fancyrefboxlabelprefix}{box}
\fancyrefaddcaptions{english}{%
  \providecommand*{\frefboxname}{Box}%
  \providecommand*{\Frefboxname}{Box}%
}
\frefformat{plain}{\fancyrefboxlabelprefix}{\frefboxname\fancyrefdefaultspacing#1}
\Frefformat{plain}{\fancyrefboxlabelprefix}{\Frefboxname\fancyrefdefaultspacing#1}

%%
% Prints argument within hanging parentheses (i.e., parentheses that take
% up no horizontal space).  Useful in tabular environments.
\newcommand{\hangp}[1]{\makebox[0pt][r]{(}#1\makebox[0pt][l]{)}}

%% 
% Prints an asterisk that takes up no horizontal space.
% Useful in tabular environments.
\newcommand{\hangstar}{\makebox[0pt][l]{*}}

%%
% Prints a trailing space in a smart way.
\usepackage{xspace}
\usepackage{xstring}

%%
% Some shortcuts for Tufte's book titles.  The lowercase commands will
% produce the initials of the book title in italics.  The all-caps commands
% will print out the full title of the book in italics.
\newcommand{\vdqi}{\textit{VDQI}\xspace}
\newcommand{\ei}{\textit{EI}\xspace}
\newcommand{\ve}{\textit{VE}\xspace}
\newcommand{\be}{\textit{BE}\xspace}
\newcommand{\VDQI}{\textit{The Visual Display of Quantitative Information}\xspace}
\newcommand{\EI}{\textit{Envisioning Information}\xspace}
\newcommand{\VE}{\textit{Visual Explanations}\xspace}
\newcommand{\BE}{\textit{Beautiful Evidence}\xspace}

\newcommand{\TL}{Tufte-\LaTeX\xspace}

% Prints the month name (e.g., January) and the year (e.g., 2008)
\newcommand{\monthyear}{%
  \ifcase\month\or January\or February\or March\or April\or May\or June\or
  July\or August\or September\or October\or November\or
  December\fi\space\number\year
}


\newcommand{\urlwhitespacereplace}[1]{\StrSubstitute{#1}{ }{_}[\wpLink]}

\newcommand{\wikipedialink}[1]{http://en.wikipedia.org/wiki/#1}% needs \wpLink now

\newcommand{\anonymouswikipedialink}[1]{\urlwhitespacereplace{#1}\href{\wikipedialink{\wpLink}}{Wikipedia}}

\newcommand{\Wikiref}[1]{\urlwhitespacereplace{#1}\href{\wikipedialink{\wpLink}}{#1}}

% Prints an epigraph and speaker in sans serif, all-caps type.
\newcommand{\openepigraph}[2]{%
  %\sffamily\fontsize{14}{16}\selectfont
  \begin{fullwidth}
  \sffamily\large
  \begin{doublespace}
  \noindent\allcaps{#1}\\% epigraph
  \noindent\allcaps{#2}% author
  \end{doublespace}
  \end{fullwidth}
}

% Inserts a blank page
\newcommand{\blankpage}{\newpage\hbox{}\thispagestyle{empty}\newpage}

\usepackage{units}

% Typesets the font size, leading, and measure in the form of 10/12x26 pc.
\newcommand{\measure}[3]{#1/#2$\times$\unit[#3]{pc}}

% Macros for typesetting the documentation
\newcommand{\hlred}[1]{\textcolor{Maroon}{#1}}% prints in red
\newcommand{\hangleft}[1]{\makebox[0pt][r]{#1}}
\newcommand{\hairsp}{\hspace{1pt}}% hair space
\newcommand{\hquad}{\hskip0.5em\relax}% half quad space
\newcommand{\TODO}{\textcolor{red}{\bf TODO!}\xspace}
\newcommand{\na}{\quad--}% used in tables for N/A cells
\providecommand{\XeLaTeX}{X\lower.5ex\hbox{\kern-0.15em\reflectbox{E}}\kern-0.1em\LaTeX}
\newcommand{\tXeLaTeX}{\XeLaTeX\index{XeLaTeX@\protect\XeLaTeX}}
% \index{\texttt{\textbackslash xyz}@\hangleft{\texttt{\textbackslash}}\texttt{xyz}}
\newcommand{\tuftebs}{\symbol{'134}}% a backslash in tt type in OT1/T1
\newcommand{\doccmdnoindex}[2][]{\texttt{\tuftebs#2}}% command name -- adds backslash automatically (and doesn't add cmd to the index)
\newcommand{\doccmddef}[2][]{%
  \hlred{\texttt{\tuftebs#2}}\label{cmd:#2}%
  \ifthenelse{\isempty{#1}}%
    {% add the command to the index
      \index{#2 command@\protect\hangleft{\texttt{\tuftebs}}\texttt{#2}}% command name
    }%
    {% add the command and package to the index
      \index{#2 command@\protect\hangleft{\texttt{\tuftebs}}\texttt{#2} (\texttt{#1} package)}% command name
      \index{#1 package@\texttt{#1} package}\index{packages!#1@\texttt{#1}}% package name
    }%
}% command name -- adds backslash automatically
\newcommand{\doccmd}[2][]{%
  \texttt{\tuftebs#2}%
  \ifthenelse{\isempty{#1}}%
    {% add the command to the index
      \index{#2 command@\protect\hangleft{\texttt{\tuftebs}}\texttt{#2}}% command name
    }%
    {% add the command and package to the index
      \index{#2 command@\protect\hangleft{\texttt{\tuftebs}}\texttt{#2} (\texttt{#1} package)}% command name
      \index{#1 package@\texttt{#1} package}\index{packages!#1@\texttt{#1}}% package name
    }%
}% command name -- adds backslash automatically
\newcommand{\docopt}[1]{\ensuremath{\langle}\textrm{\textit{#1}}\ensuremath{\rangle}}% optional command argument
\newcommand{\docarg}[1]{\textrm{\textit{#1}}}% (required) command argument
\newenvironment{docspec}{\begin{quotation}\ttfamily\parskip0pt\parindent0pt\ignorespaces}{\end{quotation}}% command specification environment
\newcommand{\docenv}[1]{\texttt{#1}\index{#1 environment@\texttt{#1} environment}\index{environments!#1@\texttt{#1}}}% environment name
\newcommand{\docenvdef}[1]{\hlred{\texttt{#1}}\label{env:#1}\index{#1 environment@\texttt{#1} environment}\index{environments!#1@\texttt{#1}}}% environment name
\newcommand{\docpkg}[1]{\texttt{#1}\index{#1 package@\texttt{#1} package}\index{packages!#1@\texttt{#1}}}% package name
\newcommand{\doccls}[1]{\texttt{#1}}% document class name
\newcommand{\docclsopt}[1]{\texttt{#1}\index{#1 class option@\texttt{#1} class option}\index{class options!#1@\texttt{#1}}}% document class option name
\newcommand{\docclsoptdef}[1]{\hlred{\texttt{#1}}\label{clsopt:#1}\index{#1 class option@\texttt{#1} class option}\index{class options!#1@\texttt{#1}}}% document class option name defined
\newcommand{\docmsg}[2]{\bigskip\begin{fullwidth}\noindent\ttfamily#1\end{fullwidth}\medskip\par\noindent#2}
\newcommand{\docfilehook}[2]{\texttt{#1}\index{file hooks!#2}\index{#1@\texttt{#1}}}
\newcommand{\doccounter}[1]{\texttt{#1}\index{#1 counter@\texttt{#1} counter}}

\newcommand{\studyq}[1]{\marginnote{Q: #1}}

\hypersetup{colorlinks}% uncomment this line if you prefer colored hyperlinks (e.g., for onscreen viewing)

% Generates the index
\usepackage{makeidx}
\makeindex

\setcounter{tocdepth}{3}
\setcounter{secnumdepth}{3}

%%%%%%%%%%%%%%%%%%%%%%%%%%%%%%%%%%%%%%%%%%%%%%%%%%%%%%%%%%%%%%
% custom commands

\newtheorem{theorem}{\color{pastel-blue}Theorem}[section]
\newtheorem{lemma}[theorem]{\color{pastel-blue}Lemma}
\newtheorem{proposition}[theorem]{\color{pastel-blue}Proposition}
\newtheorem{corollary}[theorem]{\color{pastel-blue}Corollary}

\newenvironment{proof}[1][Proof]{\begin{trivlist}
\item[\hskip \labelsep {\bfseries #1}]}{\end{trivlist}}
\newenvironment{definition}[1][Definition]{\begin{trivlist}
\item[\hskip \labelsep {\bfseries #1}]}{\end{trivlist}}
\newenvironment{example}[1][Example]{\begin{trivlist}
\item[\hskip \labelsep {\bfseries #1}]}{\end{trivlist}}
\newenvironment{remark}[1][Remark]{\begin{trivlist}
\item[\hskip \labelsep {\bfseries #1}]}{\end{trivlist}}

\hyphenpenalty=5000

% more pastel ones
\xdefinecolor{pastel-red}{rgb}{0.77,0.31,0.32}
\xdefinecolor{pastel-green}{rgb}{0.33,0.66,0.41}
\definecolor{pastel-blue}{rgb}{0.30,0.45,0.69} % crayola blue
\definecolor{gray}{rgb}{0.2,0.2,0.2} % dark gray

\xdefinecolor{orange}{rgb}{1,0.45,0}
\xdefinecolor{green}{rgb}{0,0.35,0}
\definecolor{blue}{rgb}{0.12,0.46,0.99} % crayola blue
\definecolor{gray}{rgb}{0.2,0.2,0.2} % dark gray

\xdefinecolor{cerulean}{rgb}{0.01,0.48,0.65}
\xdefinecolor{ust-blue}{rgb}{0,0.20,0.47}
\xdefinecolor{ust-mustard}{rgb}{0.67,0.52,0.13}

%\newcommand\comment[1]{{\color{red}#1}}

\newcommand{\dy}{\partial}
\newcommand{\ddy}[2]{\frac{\dy#1}{\dy#2}}

\newcommand{\ex}{\mathrm{e}}
\newcommand{\zi}{{\rm i}}

\newcommand\Real{\mbox{Re}} % cf plain TeX's \Re and Reynolds number
\newcommand\Imag{\mbox{Im}} % cf plain TeX's \Im

\newcommand{\zbar}{{\overline{z}}}

\newcommand\Def[1]{\textbf{#1}}

\newcommand{\qed}{\hfill$\blacksquare$}
\newcommand{\qedwhite}{\hfill \ensuremath{\Box}}

%%%%%%%%%%%%%%%%%%%%%%%%%%%%%%%%%%%%%%%%%%%%%%%%%%%%%%%%%%%%%%
% some extra formatting (hacked from Patrick Farrell's notes)
%  https://courses.maths.ox.ac.uk/node/view_material/4915
%

% chapter format
\titleformat{\chapter}%
  {\huge\rmfamily\itshape\color{pastel-red}}% format applied to label+text
  {\llap{\colorbox{pastel-red}{\parbox{1.5cm}{\hfill\itshape\huge\color{white}\thechapter}}}}% label
  {1em}% horizontal separation between label and title body
  {}% before the title body
  []% after the title body

% section format
\titleformat{\section}%
  {\normalfont\Large\itshape\color{pastel-green}}% format applied to label+text
  {\llap{\colorbox{pastel-green}{\parbox{1.5cm}{\hfill\color{white}\thesection}}}}% label
  {1em}% horizontal separation between label and title body
  {}% before the title body
  []% after the title body

% subsection format
\titleformat{\subsection}%
  {\normalfont\large\itshape\color{pastel-blue}}% format applied to label+text
  {\llap{\colorbox{pastel-blue}{\parbox{1.5cm}{\hfill\color{white}\thesubsection}}}}% label
  {1em}% horizontal separation between label and title body
  {}% before the title body
  []% after the title body

%%%%%%%%%%%%%%%%%%%%%%%%%%%%%%%%%%%%%%%%%%%%%%%%%%%%%%%%%%%%%%%%%%%%%%%%%%%%%%%%

\begin{document}

% Front matter
%\frontmatter

% r.3 full title page
%\maketitle

% v.4 copyright page

\chapter*{}

\begin{fullwidth}

\par \begin{center}{\Huge Linera Algebra 2H}\end{center}

\vspace*{5mm}

\par \begin{center}{\Large typed up by B. S. H. Mithrandir}\end{center}

\vspace*{5mm}

\begin{itemize}
  \item \textit{Last compiled: \monthyear}
  \item Blended from notes of C. Keaton and J. R. Parker, Durham
  \item This was part of the Durham core second year modules. Involves
  introduction to vector spaces, linear maps, and matrices as linear maps
  \item I have personally changed the ordering quite a bit, since the ordering
  in my original I have from the original notes is obscenely jumpy and makes no
  logical sense to me...(if I were teaching this I would probably have matrices
  before linear maps)
\end{itemize}

\par

\par Licensed under the Apache License, Version 2.0 (the ``License''); you may
not use this file except in compliance with the License. You may obtain a copy
of the License at \url{http://www.apache.org/licenses/LICENSE-2.0}. Unless
required by applicable law or agreed to in writing, software distributed under
the License is distributed on an \smallcaps{``AS IS'' BASIS, WITHOUT WARRANTIES
OR CONDITIONS OF ANY KIND}, either express or implied. See the License for the
specific language governing permissions and limitations under the License.
\end{fullwidth}


%===============================================================================

\chapter{Vector spaces}

Recall that a \textbf{field} $\mathbb{F}$ equipped with an addition $+$ and a
multiplication $\times$ operation satisfies the following axioms:
\begin{itemize}
  \item operations are associative
  \item operations are commutative
  \item there exists additive and multiplicative identity
  \item there exists additive and multiplicative inverse (except for the
  additive identity)
  \item multiplication is distributive over addition
\end{itemize}

The usual rationals $\mathbb{Q}$, reals $\mathbb{R}$ and complex numbers
$\mathbb{C}$ are common fields.

%-------------------------------------------------------------------------------

\section{Bases}

We say the vectors $\{\boldsymbol{v}_i\}$ are \textbf{linearly independent} if
\begin{equation}
  \lambda_1 \boldsymbol{v}_1 + \ldots + \lambda_n \boldsymbol{v}_n = 0
\end{equation}
only has trivial solutions $\lambda_i = 0$ for all $i$, if $\boldsymbol{v}_i
\neq \boldsymbol{v}_j$ for $i\neq j$.

(examples)

The \textbf{span} of $\{\boldsymbol{v}_i\}$ is defines as the set containing all
linear combinations of the \textbf{spanning set} $\{\boldsymbol{v}_i\}$, i.e.
\begin{equation}
  \mbox{Span}(\{\boldsymbol{v}_i\}) = \left\{ \sum_{i=1}^n \lambda_i \boldsymbol{v}_i \ |\ \lambda_i \in \mathbb{F} \right\}.
\end{equation}

The set $\{\boldsymbol{v}_i\}$ is a \textbf{basis} of some vector space $U$ if
the spanning set is linearly independent, and spans $U$.

%-------------------------------------------------------------------------------

\section{Sums and direct sums}

%-------------------------------------------------------------------------------

\section{Inner products}

%-------------------------------------------------------------------------------

\section{Gram--Schmidt orthonomalisation}

%-------------------------------------------------------------------------------

\section{Orthogonal projection}

%===============================================================================

\chapter{Linear maps}

%-------------------------------------------------------------------------------

\section{Spaces of linear maps}

%-------------------------------------------------------------------------------

\section{Dual space}

%===============================================================================

\section{Matrices}

Mostly going to assume we are working with matrices defined over the reals,
unless stated otherwise.

%-------------------------------------------------------------------------------

\section{Elementary matrices}

%-------------------------------------------------------------------------------

\section{Rank}

Let $\mathsf{A}$ be a $m\times n$ matrix. Then the \textbf{row rank} of
$\mathsf{A}$ is the dimension of the subspaces spanned by the row vectors of
$\mathsf{A}$. The \textbf{column rank} is defined similarly.

\begin{theorem}
  For every matrix, the row rank is equal to the column rank (so we only need to
  talk about \emph{rank}).
\end{theorem}

\begin{proof}
  Think of $\mathsf{A}$ as a linear map $\theta: \mathbb{R}^n \to \mathbb{R}^m$,
  so there exists non-singular $\mathsf{X}$ and $\mathsf{Y}$ where
  \begin{equation*}
    \mathsf{B} = \mathsf{XAY} = \begin{pmatrix}\mathsf{I}_r & 0 \\ 0 & 0 \end{pmatrix},
  \end{equation*}
  where $r$ is the row rank of $\theta$. The transpose of $\mathsf{B}$ has the
  same rank, although that is now the column rank of $\mathsf{B}^T$.
\end{proof}

We define the \textbf{nullity} of $\mathsf{A}$ to be the dimension of the
kernel, i.e., the dimension of the solution set $\mathsf{A}\boldsymbol{x} =
\boldsymbol{0}$.

\begin{lemma}
  The following are equivalent if $\mathsf{A}$ is a square $n\times n$ matrix:
  \begin{enumerate}
    \item $\mathsf{A}$ is invertible,
    \item $\mbox{rank}(\mathsf{A}) = n$,
    \item $\mbox{null}(\mathsf{A}) = 0$.
  \end{enumerate}
\end{lemma}

\begin{proof}
  If $\mathsf{A}$ is invertible then the corresponding linear map $\theta$ is an
  isomorphism, and we have the result by a previous corollary. {\color{red}reference back}
\end{proof}

%-------------------------------------------------------------------------------

\section{Determinants}

%-------------------------------------------------------------------------------

\section{Adjugate}

%-------------------------------------------------------------------------------

\section{Eigenvalues and eigenvectors}

Let $V$ be a vector space, and $\theta:V \to V$ be a linear map. Then
$\boldsymbol{v}\in V-\{\boldsymbol{0}\}$ is an \textbf{eigenvector} of $\theta$
if
\begin{equation}
  \theta(\boldsymbol{v}) = \lambda \boldsymbol{v},
\end{equation}
where $\lambda$ is the corresponding \textbf{eigenvalue}. The subset
\begin{equation}
  V_{\lambda} = \{\boldsymbol{v}\in V \ |\ \theta(\boldsymbol{v}) = \lambda \boldsymbol{v}\}
\end{equation}
is a vector space called the \textbf{eigenspace} corresponding to the eigenvalue
$\lambda$.

Clearly if $\mathsf{A}$ is the corresponding matrix to $\theta$ then the
eigenvalues and eigenvectors would correspond accordingly.

Two $n\times n$ matrices $\mathsf{A}$ and $\mathsf{B}$ are called
\textbf{similar} if there exists some non-singular $\mathsf{Y}$ where
$\mathsf{B} = \mathsf{Y}^{-1}\mathsf{AY}$.

\begin{theorem}
  If $\mathsf{A}$ corresponds to the linear map $\theta:V\to V$, then the roots
  of the \textbf{characteristic polynomial} $|\lambda\mathsf{I} - \mathsf{A}| =
  0$ are the eigenvalues of $\theta$. This is independent of the basis of $V$.
\end{theorem}

%-------------------------------------------------------------------------------

\section{Cayley--Hamilton theorem}

%-------------------------------------------------------------------------------

\section{Symmetric and Hermitian matrices}

%-------------------------------------------------------------------------------

\section{Orthogonal matrices}

%-------------------------------------------------------------------------------

\section{Symmetric matrices}

%-------------------------------------------------------------------------------

\section{Rotations}

%-------------------------------------------------------------------------------

\section{Quadratic forms}

%-------------------------------------------------------------------------------

\section{Symmetric and self-adjoint maps}

%-------------------------------------------------------------------------------

\section{Reflections}

%-------------------------------------------------------------------------------

\section{The operator $\mathrm{d}/\mathrm{d}x^2$}

%-------------------------------------------------------------------------------

\section{Jordan normal form}

%===============================================================================

%%%%%%%%%%%%%%%%%%%%%%%%%%%%%%%%%%%%%%%%%

% r.5 contents
%\tableofcontents

%\listoffigures

%\listoftables

% r.7 dedication
%\cleardoublepage
%~\vfill
%\begin{doublespace}
%\noindent\fontsize{18}{22}\selectfont\itshape
%\nohyphenation
%Dedicated to those who appreciate \LaTeX{} 
%and the work of \mbox{Edward R.~Tufte} 
%and \mbox{Donald E.~Knuth}.
%\end{doublespace}
%\vfill

% r.9 introduction
% \cleardoublepage

%%%%%%%%%%%%%%%%%%%%%%%%%%%%%%%%%%%%%%%%%
% actual useful crap (normal chapters)
\mainmatter

%\part{Basics (?)}


%\backmatter

%\bibliography{refs}
\bibliographystyle{plainnat}

%\printindex

\end{document}

