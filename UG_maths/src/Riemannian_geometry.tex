\documentclass[letter-paper]{tufte-book}

%%
% Book metadata
\title{Riemannian geometry 4H}
\author[]{B. S. H. Mithrandir}
%\publisher{Research Institute of Valinor}

%%
% If they're installed, use Bergamo and Chantilly from www.fontsite.com.
% They're clones of Bembo and Gill Sans, respectively.
\IfFileExists{bergamo.sty}{\usepackage[osf]{bergamo}}{}% Bembo
\IfFileExists{chantill.sty}{\usepackage{chantill}}{}% Gill Sans

%\usepackage{microtype}
\usepackage{amssymb}
\usepackage{amsmath}
%%
% For nicely typeset tabular material
\usepackage{booktabs}

%% overunder braces
\usepackage{oubraces}

%% 
\usepackage{xcolor}
\usepackage{tcolorbox}

\newtcolorbox[auto counter,number within=section]{derivbox}[2][]{colback=TealBlue!5!white,colframe=TealBlue,title=Box \thetcbcounter:\ #2,#1}                                                          

\makeatletter
\@openrightfalse
\makeatother

%%
% For graphics / images
\usepackage{graphicx}
\setkeys{Gin}{width=\linewidth,totalheight=\textheight,keepaspectratio}
\graphicspath{{figs/}}

\usepackage{tikz-cd}

% The fancyvrb package lets us customize the formatting of verbatim
% environments.  We use a slightly smaller font.
\usepackage{fancyvrb}
\fvset{fontsize=\normalsize}

\usepackage[plain]{fancyref}
\newcommand*{\fancyrefboxlabelprefix}{box}
\fancyrefaddcaptions{english}{%
  \providecommand*{\frefboxname}{Box}%
  \providecommand*{\Frefboxname}{Box}%
}
\frefformat{plain}{\fancyrefboxlabelprefix}{\frefboxname\fancyrefdefaultspacing#1}
\Frefformat{plain}{\fancyrefboxlabelprefix}{\Frefboxname\fancyrefdefaultspacing#1}

%%
% Prints argument within hanging parentheses (i.e., parentheses that take
% up no horizontal space).  Useful in tabular environments.
\newcommand{\hangp}[1]{\makebox[0pt][r]{(}#1\makebox[0pt][l]{)}}

%% 
% Prints an asterisk that takes up no horizontal space.
% Useful in tabular environments.
\newcommand{\hangstar}{\makebox[0pt][l]{*}}

%%
% Prints a trailing space in a smart way.
\usepackage{xspace}
\usepackage{xstring}

%%
% Some shortcuts for Tufte's book titles.  The lowercase commands will
% produce the initials of the book title in italics.  The all-caps commands
% will print out the full title of the book in italics.
\newcommand{\vdqi}{\textit{VDQI}\xspace}
\newcommand{\ei}{\textit{EI}\xspace}
\newcommand{\ve}{\textit{VE}\xspace}
\newcommand{\be}{\textit{BE}\xspace}
\newcommand{\VDQI}{\textit{The Visual Display of Quantitative Information}\xspace}
\newcommand{\EI}{\textit{Envisioning Information}\xspace}
\newcommand{\VE}{\textit{Visual Explanations}\xspace}
\newcommand{\BE}{\textit{Beautiful Evidence}\xspace}

\newcommand{\TL}{Tufte-\LaTeX\xspace}

% Prints the month name (e.g., January) and the year (e.g., 2008)
\newcommand{\monthyear}{%
  \ifcase\month\or January\or February\or March\or April\or May\or June\or
  July\or August\or September\or October\or November\or
  December\fi\space\number\year
}


\newcommand{\urlwhitespacereplace}[1]{\StrSubstitute{#1}{ }{_}[\wpLink]}

\newcommand{\wikipedialink}[1]{http://en.wikipedia.org/wiki/#1}% needs \wpLink now

\newcommand{\anonymouswikipedialink}[1]{\urlwhitespacereplace{#1}\href{\wikipedialink{\wpLink}}{Wikipedia}}

\newcommand{\Wikiref}[1]{\urlwhitespacereplace{#1}\href{\wikipedialink{\wpLink}}{#1}}

% Prints an epigraph and speaker in sans serif, all-caps type.
\newcommand{\openepigraph}[2]{%
  %\sffamily\fontsize{14}{16}\selectfont
  \begin{fullwidth}
  \sffamily\large
  \begin{doublespace}
  \noindent\allcaps{#1}\\% epigraph
  \noindent\allcaps{#2}% author
  \end{doublespace}
  \end{fullwidth}
}

% Inserts a blank page
\newcommand{\blankpage}{\newpage\hbox{}\thispagestyle{empty}\newpage}

\usepackage{units}

% Typesets the font size, leading, and measure in the form of 10/12x26 pc.
\newcommand{\measure}[3]{#1/#2$\times$\unit[#3]{pc}}

% Macros for typesetting the documentation
\newcommand{\hlred}[1]{\textcolor{Maroon}{#1}}% prints in red
\newcommand{\hangleft}[1]{\makebox[0pt][r]{#1}}
\newcommand{\hairsp}{\hspace{1pt}}% hair space
\newcommand{\hquad}{\hskip0.5em\relax}% half quad space
\newcommand{\TODO}{\textcolor{red}{\bf TODO!}\xspace}
\newcommand{\na}{\quad--}% used in tables for N/A cells
\providecommand{\XeLaTeX}{X\lower.5ex\hbox{\kern-0.15em\reflectbox{E}}\kern-0.1em\LaTeX}
\newcommand{\tXeLaTeX}{\XeLaTeX\index{XeLaTeX@\protect\XeLaTeX}}
% \index{\texttt{\textbackslash xyz}@\hangleft{\texttt{\textbackslash}}\texttt{xyz}}
\newcommand{\tuftebs}{\symbol{'134}}% a backslash in tt type in OT1/T1
\newcommand{\doccmdnoindex}[2][]{\texttt{\tuftebs#2}}% command name -- adds backslash automatically (and doesn't add cmd to the index)
\newcommand{\doccmddef}[2][]{%
  \hlred{\texttt{\tuftebs#2}}\label{cmd:#2}%
  \ifthenelse{\isempty{#1}}%
    {% add the command to the index
      \index{#2 command@\protect\hangleft{\texttt{\tuftebs}}\texttt{#2}}% command name
    }%
    {% add the command and package to the index
      \index{#2 command@\protect\hangleft{\texttt{\tuftebs}}\texttt{#2} (\texttt{#1} package)}% command name
      \index{#1 package@\texttt{#1} package}\index{packages!#1@\texttt{#1}}% package name
    }%
}% command name -- adds backslash automatically
\newcommand{\doccmd}[2][]{%
  \texttt{\tuftebs#2}%
  \ifthenelse{\isempty{#1}}%
    {% add the command to the index
      \index{#2 command@\protect\hangleft{\texttt{\tuftebs}}\texttt{#2}}% command name
    }%
    {% add the command and package to the index
      \index{#2 command@\protect\hangleft{\texttt{\tuftebs}}\texttt{#2} (\texttt{#1} package)}% command name
      \index{#1 package@\texttt{#1} package}\index{packages!#1@\texttt{#1}}% package name
    }%
}% command name -- adds backslash automatically
\newcommand{\docopt}[1]{\ensuremath{\langle}\textrm{\textit{#1}}\ensuremath{\rangle}}% optional command argument
\newcommand{\docarg}[1]{\textrm{\textit{#1}}}% (required) command argument
\newenvironment{docspec}{\begin{quotation}\ttfamily\parskip0pt\parindent0pt\ignorespaces}{\end{quotation}}% command specification environment
\newcommand{\docenv}[1]{\texttt{#1}\index{#1 environment@\texttt{#1} environment}\index{environments!#1@\texttt{#1}}}% environment name
\newcommand{\docenvdef}[1]{\hlred{\texttt{#1}}\label{env:#1}\index{#1 environment@\texttt{#1} environment}\index{environments!#1@\texttt{#1}}}% environment name
\newcommand{\docpkg}[1]{\texttt{#1}\index{#1 package@\texttt{#1} package}\index{packages!#1@\texttt{#1}}}% package name
\newcommand{\doccls}[1]{\texttt{#1}}% document class name
\newcommand{\docclsopt}[1]{\texttt{#1}\index{#1 class option@\texttt{#1} class option}\index{class options!#1@\texttt{#1}}}% document class option name
\newcommand{\docclsoptdef}[1]{\hlred{\texttt{#1}}\label{clsopt:#1}\index{#1 class option@\texttt{#1} class option}\index{class options!#1@\texttt{#1}}}% document class option name defined
\newcommand{\docmsg}[2]{\bigskip\begin{fullwidth}\noindent\ttfamily#1\end{fullwidth}\medskip\par\noindent#2}
\newcommand{\docfilehook}[2]{\texttt{#1}\index{file hooks!#2}\index{#1@\texttt{#1}}}
\newcommand{\doccounter}[1]{\texttt{#1}\index{#1 counter@\texttt{#1} counter}}

\newcommand{\studyq}[1]{\marginnote{Q: #1}}

\hypersetup{colorlinks}% uncomment this line if you prefer colored hyperlinks (e.g., for onscreen viewing)

% Generates the index
\usepackage{makeidx}
\makeindex

\setcounter{tocdepth}{3}
\setcounter{secnumdepth}{3}

%%%%%%%%%%%%%%%%%%%%%%%%%%%%%%%%%%%%%%%%%%%%%%%%%%%%%%%%%%%%%%
% custom commands

\newtheorem{theorem}{\color{pastel-blue}Theorem}[section]
\newtheorem{lemma}[theorem]{\color{pastel-blue}Lemma}
\newtheorem{proposition}[theorem]{\color{pastel-blue}Proposition}
\newtheorem{corollary}[theorem]{\color{pastel-blue}Corollary}

\newenvironment{proof}[1][Proof]{\begin{trivlist}
\item[\hskip \labelsep {\bfseries #1}]}{\end{trivlist}}
\newenvironment{definition}[1][Definition]{\begin{trivlist}
\item[\hskip \labelsep {\bfseries #1}]}{\end{trivlist}}
\newenvironment{example}[1][Example]{\begin{trivlist}
\item[\hskip \labelsep {\bfseries #1}]}{\end{trivlist}}
\newenvironment{remark}[1][Remark]{\begin{trivlist}
\item[\hskip \labelsep {\bfseries #1}]}{\end{trivlist}}

\hyphenpenalty=5000

% more pastel ones
\xdefinecolor{pastel-red}{rgb}{0.77,0.31,0.32}
\xdefinecolor{pastel-green}{rgb}{0.33,0.66,0.41}
\definecolor{pastel-blue}{rgb}{0.30,0.45,0.69} % crayola blue
\definecolor{gray}{rgb}{0.2,0.2,0.2} % dark gray

\xdefinecolor{orange}{rgb}{1,0.45,0}
\xdefinecolor{green}{rgb}{0,0.35,0}
\definecolor{blue}{rgb}{0.12,0.46,0.99} % crayola blue
\definecolor{gray}{rgb}{0.2,0.2,0.2} % dark gray

\xdefinecolor{cerulean}{rgb}{0.01,0.48,0.65}
\xdefinecolor{ust-blue}{rgb}{0,0.20,0.47}
\xdefinecolor{ust-mustard}{rgb}{0.67,0.52,0.13}

%\newcommand\comment[1]{{\color{red}#1}}

\newcommand{\dy}{\partial}
\newcommand{\ddy}[2]{\frac{\dy#1}{\dy#2}}

\newcommand{\ex}{\mathrm{e}}
\newcommand{\zi}{{\rm i}}

\newcommand\Real{\mbox{Re}} % cf plain TeX's \Re and Reynolds number
\newcommand\Imag{\mbox{Im}} % cf plain TeX's \Im

\newcommand\Def[1]{\textbf{#1}}

\newcommand{\qed}{\hfill$\blacksquare$}
\newcommand{\qedwhite}{\hfill \ensuremath{\Box}}

\newcommand{\highlight}[1]{\mathchoice%
  {\colorbox{black!10}{$\displaystyle#1$}}%
  {\colorbox{black!10}{$\textstyle#1$}}%
  {\colorbox{black!10}{$\scriptstyle#1$}}%
  {\colorbox{black!10}{$\scriptscriptstyle#1$}}}%

%%%%%%%%%%%%%%%%%%%%%%%%%%%%%%%%%%%%%%%%%%%%%%%%%%%%%%%%%%%%%%
% some extra formatting (hacked from Patrick Farrell's notes)
%  https://courses.maths.ox.ac.uk/node/view_material/4915
%

% chapter format
\titleformat{\chapter}%
  {\huge\rmfamily\itshape\color{pastel-red}}% format applied to label+text
  {\llap{\colorbox{pastel-red}{\parbox{1.5cm}{\hfill\itshape\huge\color{white}\thechapter}}}}% label
  {1em}% horizontal separation between label and title body
  {}% before the title body
  []% after the title body

% section format
\titleformat{\section}%
  {\normalfont\Large\itshape\color{pastel-green}}% format applied to label+text
  {\llap{\colorbox{pastel-green}{\parbox{1.5cm}{\hfill\color{white}\thesection}}}}% label
  {1em}% horizontal separation between label and title body
  {}% before the title body
  []% after the title body

% subsection format
\titleformat{\subsection}%
  {\normalfont\large\itshape\color{pastel-blue}}% format applied to label+text
  {\llap{\colorbox{pastel-blue}{\parbox{1.5cm}{\hfill\color{white}\thesubsection}}}}% label
  {1em}% horizontal separation between label and title body
  {}% before the title body
  []% after the title body

%%%%%%%%%%%%%%%%%%%%%%%%%%%%%%%%%%%%%%%%%%%%%%%%%%%%%%%%%%%%%%%%%%%%%%%%%%%%%%%%

\begin{document}

% Front matter
%\frontmatter

% r.3 full title page
%\maketitle

% v.4 copyright page

\chapter*{}

\begin{fullwidth}

\par \begin{center}{\Huge Riemannian geometry 4H}\end{center}

\vspace*{5mm}

\par \begin{center}{\Large typed up by B. S. H. Mithrandir}\end{center}

\vspace*{5mm}

\begin{itemize}
  \item \textit{Last compiled: \monthyear}
  \item Adapted from notes of (I think?) W. Klingenberg, Durham
  \item This was part of the Riemannian Geometric 4H module elective, as a follow on to the Differential geometry 3H course. Probably would help having gone through the Analysis 3H course also.
  \item[]
  \item \TODO diagrams
\end{itemize}

\par

\par Licensed under the Apache License, Version 2.0 (the ``License''); you may not
use this file except in compliance with the License. You may obtain a copy
of the License at \url{http://www.apache.org/licenses/LICENSE-2.0}. Unless
required by applicable law or agreed to in writing, software distributed
under the License is distributed on an \smallcaps{``AS IS'' BASIS, WITHOUT
WARRANTIES OR CONDITIONS OF ANY KIND}, either express or implied. See the
License for the specific language governing permissions and limitations
under the License.
\end{fullwidth}


%===============================================================================

\chapter{Differentiable manifolds}

Recall that a $k$-manifold $M$ is one which, locally, looked like open sets of $\mathbb{R}^k$. The manifold is build up by patching together these local sets.

\begin{example}
  \begin{enumerate}
    \item A surface of revoluation $M$ with $\gamma : [a, b] \to \mathbb {R}^3$ with $\gamma(t) = (\gamma_1(t), 0, \gamma_3(t))$ (assuming $\gamma'(t) \neq 0$ for all $t$ and $\gamma$ is simple) is given by
    \begin{align*}
      M &= \{ (\gamma_1(t), \gamma_3(t)\cos\theta, \gamma_3(t)\sin\theta)\ |\ t\in[a,b],\ \theta \in [0, 2\pi]\}\\
        &= \{ (x_1, x_2, x_3)\ |\ \exists t\in(a,b),\ x_1 = \gamma_1(t),\ x_2^2 + x_3^2 = \gamma_3^2(t)\}.
    \end{align*}
    
    \item The matrix group $\mbox{SL}(n, \mathbb{R})$, $\mbox{O}(n)$ and $\mbox{SO}(n)$ can be considered as submanifolds of the ambient space $\mathbb{R}^{n^2} = M(n, \mathbb{R}$.
    
    \item The real projective plane/space $\mathbb{R}\mbox{P}^n$ is given by the set of all lines through the origin in $\mathbb{R}^{n+1}$. It can be identified with the quotient group $S^n / \sim$, where $p \sim q$ iff $p = \pm q$.
  \end{enumerate}
\end{example}

Recall that for an $n$-dimensional subspace $X \subset \mathbb{R}^n$, $x \in X$ is called an \textbf{inner point} of iff there exists an $n$-dimensional open $U \subset X$ with $x\in U$. $X$ is \textbf{open} if every $x \in X$ is an inner point.

Let $M$ be a set. A collection $(U_\alpha, \phi_\alpha)$ with index $\alpha$ is called an \textbf{atlas} of $M$ is
\begin{enumerate}
  \item $U_\alpha \in M$ and $\bigcup_\alpha U_\alpha = M$ (the collection of $U_\alpha$ covers $M$),
  
  \item $\phi_\alpha : U_\alpha \to V_\alpha \subset \mathbb{R}^n$ are bijective maps with open $V_\alpha$, and for each $\alpha, \beta$, $\phi_\alpha(U_\alpha \cap U_\beta)$ is open,
  
  \item the co-ordinate changes $\phi_\beta \circ \phi_\alpha^{-1} : \phi_\alpha(U_\alpha \cap U_\beta) \to \phi_\beta(U_\alpha \cap U_\beta)$ are smooth maps between open subsets of $\mathbb{R}^n$.
\end{enumerate}

Let $(U_\alpha, \phi_\alpha)$ be an atlas of $M$, and $X \subset M$. A point $x \in X$ is called an \textbf{inner point} of $X$ is there exists $\phi_\alpha : U_\alpha \to V_\alpha \subset \mathbb{R}^n$ such that $\phi_\alpha(x)$ is an inner point of $\phi_\alpha (X \cap U_\alpha)$ for $x \in U_\alpha$. $X$ is \textbf{open} if every $x \in X$ is an inner point.

A set $M$ is a \textbf{differentiable manifold} if it carries an atlas $(U_\alpha, \phi_\alpha)$ and has the \textbf{Hausdorff property}, i.e. for any $p, q \in M$ with $p \neq q$, there exists open $A_p, A_q \subset M$ such that for $p \in A_p$ and $q \in A_q$ where $A_p \cap A_q = \emptyset$.\marginnote{This would be the $T_2$ condition relating to topological spaces. Distinct points have disjoint neighbourhoods.}

\begin{example}
  Consider the real line with zero removed, and we take $M = (\mathbb{R} \ \{0\}) \cup \{0_+, 0_-\}$ where $0_\pm$ are points off the real line by arbitrarily near zero. We can construct the smooth charts
  \begin{equation*}
    \phi_i : (\mathbb{R} \ \{0\}) \cup \{0_i\} \to \mathbb{R}, \qquad x \mapsto \begin{cases}x, & x \neq 0_i, \\ 0, & x=0_i,\end{cases}
  \end{equation*}
  and $\{\phi_+, \phi_-\}$ can serve as an atlas, but we do not have the Hausdorff property, and $M$ is not a differentiable manifold.
\end{example}

%-------------------------------------------------------------------------------

\section{Manifolds and regular values}

Let $f : U \to \mathbb{R}^k$ be differentiable and $U$ is open, and denote
\begin{equation}
  \highlight{Df(x) = \left[\frac{\partial f_i}{\partial x_j}\right]_{ij} : \mathbb{R}^n \to \mathbb{R}^k}
\end{equation}
be the Jacobian matrix at $x \in U$. The point $x \in U$ is a \textbf{regular point} if $Df(x)$ is surjective, i.e. of rank $k$. $y \in \mathbb{R}^k$ is a \textbf{regular value} of $f$ if all $x \in f^{-1}(y)$ are regular points.

\begin{theorem}
  Let $U \subset \mathbb{R}^n$ be open, $f : U \to \mathbb{R}^k$ be differentiable, $k \leq n$, and $y \in \mathbb{R}^k$ be a regular value of $f$. Then $M = f^{-1}(y) \subset U$ is a differentiable manifold of dimension $(n-k)$. \qedwhite
\end{theorem}

\begin{theorem}[Implicit function theorem]
  For differentiable $f : \mathbb{R}^{n-k} \times \mathbb{R}^k \to \mathbb{R}^k$, $f(x^1, x^2) = y$, then we have
  \begin{equation*}
    df(x^1, x^2) = \left[\frac{\partial f}{\partial x^1}(x^1, x^2) \quad \frac{\partial f}{\partial x^2}(x^1, x^2)\right].
  \end{equation*}
  Assuming $\mbox{det}\left[\partial f / \partial x^2(x^1, x^2)\right]$ is non-zero, then there exists neighbourhoods $V_1 \subset \mathbb{R}^{n-k}$ and $V_2 \subset \mathbb{R}^k$ (with $x^1 \in V_1$ and $x^2 \in V_2$) where there is some $\psi : V_1 \to V_2$ that is differentiable, and such that $f^{-1}(y) \cap (V_1 \times V_2) = \{(x^1, \psi(x^1)\ :\ x^1 \in V_1\}$. \qedwhite
\end{theorem}

Note that for differentiable $f : \mathbb{R}^n \to \mathbb{R}^k$, by the chain rule we have
\begin{equation}
  Df(x) \cdot z = \lim_{h \to 0} \frac{f(x + hz) - f(x)}{h}, \qquad z \in \mathbb{R}^n
\end{equation}
since $Df(x) : \mathbb{R}^n \to \mathbb{R}^k$ This can be regarded as a \textbf{directional derivative in the direction $z$}.

\begin{example}
  The group $\mbox{SO}(n)$ can be considered as a manifold. Consider $\mbox{GL}^+(n) = \{A \in \mbox{M}(n, \mathbb{R})\ |\ |A| > 0\}$, which is an open set in $\mbox{M}(n, \mathbb{R}) \cong \mathbb{R}^{n^2}$, as $\mbox{det} : \mbox{M}(n, \mathbb{R}) \to \mathbb{R}$ is a continuous function. Note that $AA^T = I$ means $|A| = \pm 1$, so
  \begin{equation*}
    \mbox{SO}(n) = \mbox{O}(n) \cap \mbox{GL}^+(n).
  \end{equation*}
  Consider $f : \mbox{GL}^+(n) \to \mbox{Sym}(n) = \{C \in \mbox{M}(n, \mathbb{R})\ |\ C^T = C\} \cong \mathbb{R}^{n(n+1)/2}$, with $f(A) = AA^T - I$. Then we clearly have $f^{-1}(0) = \mbox{SO}(n)$, and observe that
  \begin{align*}
    \left.Df\right|_A B 
      &= \lim_{h \to 0} \frac{f(A + hB) - f(A)}{h}\\
      &= \lim_{h \to 0} \left[ \frac{(A + hB)^T (A + hB) - A^T A}{h} \right]\\
      &= \lim_{h \to 0} \left[ A^T B + B^T A + t B^T B \right]\\
      &= A^T B + B^T A.
  \end{align*}
  We need to check if $Df$ is surjective for all $A \in f^{-1}(0) = \mbox{SO}(n)$. For $C \in \mbox{Sym}(n)$, we have
  \begin{align*}
    \left.Df\right|_A \left(\frac{1}{2}AC \right) = \frac{1}{2} A^T (AC) + \frac{1}{2}(AC)^T A = C
  \end{align*}
  since $AA^T = I$ and $C = C^T$. Since $C$ was arbitrary, $Df$ is surjective, thus $0$ is a regular value and $\mbox{SO}(n)$ is a $n^2 - n(n+1)/2 = n(n-1)/2$ dimensional manifold.
\end{example}

%-------------------------------------------------------------------------------

\section{Differentiable maps, tangent vectors, and differentials}

%-------------------------------------------------------------------------------

\section{Lie groups}

%-------------------------------------------------------------------------------

\section{Tangent bundles, vector fields and Lie brackets}

%===============================================================================

\chapter{Riemannian manifolds}

%-------------------------------------------------------------------------------

\section{Integration on Riemannian manifolds}

%-------------------------------------------------------------------------------

\section{Riemannian manifolds as metric spaces}

%===============================================================================

\chapter{Levi-Civita connection and parallel transport}

%-------------------------------------------------------------------------------

\section{Christoffel symbols}

%-------------------------------------------------------------------------------

\section{Parallel transport}

%-------------------------------------------------------------------------------

\section{Geodesics}

%-------------------------------------------------------------------------------

\section{Geodesic flow}

%===============================================================================

\chapter{Curvature}

%-------------------------------------------------------------------------------

\section{Sectional curvature}

%-------------------------------------------------------------------------------

\section{Ricci and scalar curvature}

%-------------------------------------------------------------------------------

\section{Isometric immersions}

%-------------------------------------------------------------------------------

\section{The second fundamental form}

%-------------------------------------------------------------------------------

\section{Second variational formular for length}



%===============================================================================

%%%%%%%%%%%%%%%%%%%%%%%%%%%%%%%%%%%%%%%%%

% r.5 contents
%\tableofcontents

%\listoffigures

%\listoftables

% r.7 dedication
%\cleardoublepage
%~\vfill
%\begin{doublespace}
%\noindent\fontsize{18}{22}\selectfont\itshape
%\nohyphenation
%Dedicated to those who appreciate \LaTeX{} 
%and the work of \mbox{Edward R.~Tufte} 
%and \mbox{Donald E.~Knuth}.
%\end{doublespace}
%\vfill

% r.9 introduction
% \cleardoublepage

%%%%%%%%%%%%%%%%%%%%%%%%%%%%%%%%%%%%%%%%%
% actual useful crap (normal chapters)
\mainmatter

%\part{Basics (?)}


%\backmatter

%\bibliography{refs}
\bibliographystyle{plainnat}

%\printindex

\end{document}

