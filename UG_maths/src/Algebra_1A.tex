\documentclass[letter-paper]{tufte-book}

%%
% Book metadata
\title{Algebra 1H}
\author[]{Inusuke Shibemoto}
%\publisher{Research Institute of Valinor}

%%
% If they're installed, use Bergamo and Chantilly from www.fontsite.com.
% They're clones of Bembo and Gill Sans, respectively.
\IfFileExists{bergamo.sty}{\usepackage[osf]{bergamo}}{}% Bembo
\IfFileExists{chantill.sty}{\usepackage{chantill}}{}% Gill Sans

%\usepackage{microtype}
\usepackage{amssymb}
\usepackage{amsmath}
%%
% For nicely typeset tabular material
\usepackage{booktabs}

%% overunder braces
\usepackage{oubraces}

%% 
\usepackage{xcolor}
\usepackage{tcolorbox}

\newtcolorbox[auto counter,number within=section]{derivbox}[2][]{colback=TealBlue!5!white,colframe=TealBlue,title=Box \thetcbcounter:\ #2,#1}                                                          

\makeatletter
\@openrightfalse
\makeatother

%%
% For graphics / images
\usepackage{graphicx}
\setkeys{Gin}{width=\linewidth,totalheight=\textheight,keepaspectratio}
\graphicspath{{figs/}}

% The fancyvrb package lets us customize the formatting of verbatim
% environments.  We use a slightly smaller font.
\usepackage{fancyvrb}
\fvset{fontsize=\normalsize}

\usepackage[plain]{fancyref}
\newcommand*{\fancyrefboxlabelprefix}{box}
\fancyrefaddcaptions{english}{%
  \providecommand*{\frefboxname}{Box}%
  \providecommand*{\Frefboxname}{Box}%
}
\frefformat{plain}{\fancyrefboxlabelprefix}{\frefboxname\fancyrefdefaultspacing#1}
\Frefformat{plain}{\fancyrefboxlabelprefix}{\Frefboxname\fancyrefdefaultspacing#1}

%%
% Prints argument within hanging parentheses (i.e., parentheses that take
% up no horizontal space).  Useful in tabular environments.
\newcommand{\hangp}[1]{\makebox[0pt][r]{(}#1\makebox[0pt][l]{)}}

%% 
% Prints an asterisk that takes up no horizontal space.
% Useful in tabular environments.
\newcommand{\hangstar}{\makebox[0pt][l]{*}}

%%
% Prints a trailing space in a smart way.
\usepackage{xspace}
\usepackage{xstring}

%%
% Some shortcuts for Tufte's book titles.  The lowercase commands will
% produce the initials of the book title in italics.  The all-caps commands
% will print out the full title of the book in italics.
\newcommand{\vdqi}{\textit{VDQI}\xspace}
\newcommand{\ei}{\textit{EI}\xspace}
\newcommand{\ve}{\textit{VE}\xspace}
\newcommand{\be}{\textit{BE}\xspace}
\newcommand{\VDQI}{\textit{The Visual Display of Quantitative Information}\xspace}
\newcommand{\EI}{\textit{Envisioning Information}\xspace}
\newcommand{\VE}{\textit{Visual Explanations}\xspace}
\newcommand{\BE}{\textit{Beautiful Evidence}\xspace}

\newcommand{\TL}{Tufte-\LaTeX\xspace}

% Prints the month name (e.g., January) and the year (e.g., 2008)
\newcommand{\monthyear}{%
  \ifcase\month\or January\or February\or March\or April\or May\or June\or
  July\or August\or September\or October\or November\or
  December\fi\space\number\year
}


\newcommand{\urlwhitespacereplace}[1]{\StrSubstitute{#1}{ }{_}[\wpLink]}

\newcommand{\wikipedialink}[1]{http://en.wikipedia.org/wiki/#1}% needs \wpLink now

\newcommand{\anonymouswikipedialink}[1]{\urlwhitespacereplace{#1}\href{\wikipedialink{\wpLink}}{Wikipedia}}

\newcommand{\Wikiref}[1]{\urlwhitespacereplace{#1}\href{\wikipedialink{\wpLink}}{#1}}

% Prints an epigraph and speaker in sans serif, all-caps type.
\newcommand{\openepigraph}[2]{%
  %\sffamily\fontsize{14}{16}\selectfont
  \begin{fullwidth}
  \sffamily\large
  \begin{doublespace}
  \noindent\allcaps{#1}\\% epigraph
  \noindent\allcaps{#2}% author
  \end{doublespace}
  \end{fullwidth}
}

% Inserts a blank page
\newcommand{\blankpage}{\newpage\hbox{}\thispagestyle{empty}\newpage}

\usepackage{units}

% Typesets the font size, leading, and measure in the form of 10/12x26 pc.
\newcommand{\measure}[3]{#1/#2$\times$\unit[#3]{pc}}

% Macros for typesetting the documentation
\newcommand{\hlred}[1]{\textcolor{Maroon}{#1}}% prints in red
\newcommand{\hangleft}[1]{\makebox[0pt][r]{#1}}
\newcommand{\hairsp}{\hspace{1pt}}% hair space
\newcommand{\hquad}{\hskip0.5em\relax}% half quad space
\newcommand{\TODO}{\textcolor{red}{\bf TODO!}\xspace}
\newcommand{\na}{\quad--}% used in tables for N/A cells
\providecommand{\XeLaTeX}{X\lower.5ex\hbox{\kern-0.15em\reflectbox{E}}\kern-0.1em\LaTeX}
\newcommand{\tXeLaTeX}{\XeLaTeX\index{XeLaTeX@\protect\XeLaTeX}}
% \index{\texttt{\textbackslash xyz}@\hangleft{\texttt{\textbackslash}}\texttt{xyz}}
\newcommand{\tuftebs}{\symbol{'134}}% a backslash in tt type in OT1/T1
\newcommand{\doccmdnoindex}[2][]{\texttt{\tuftebs#2}}% command name -- adds backslash automatically (and doesn't add cmd to the index)
\newcommand{\doccmddef}[2][]{%
  \hlred{\texttt{\tuftebs#2}}\label{cmd:#2}%
  \ifthenelse{\isempty{#1}}%
    {% add the command to the index
      \index{#2 command@\protect\hangleft{\texttt{\tuftebs}}\texttt{#2}}% command name
    }%
    {% add the command and package to the index
      \index{#2 command@\protect\hangleft{\texttt{\tuftebs}}\texttt{#2} (\texttt{#1} package)}% command name
      \index{#1 package@\texttt{#1} package}\index{packages!#1@\texttt{#1}}% package name
    }%
}% command name -- adds backslash automatically
\newcommand{\doccmd}[2][]{%
  \texttt{\tuftebs#2}%
  \ifthenelse{\isempty{#1}}%
    {% add the command to the index
      \index{#2 command@\protect\hangleft{\texttt{\tuftebs}}\texttt{#2}}% command name
    }%
    {% add the command and package to the index
      \index{#2 command@\protect\hangleft{\texttt{\tuftebs}}\texttt{#2} (\texttt{#1} package)}% command name
      \index{#1 package@\texttt{#1} package}\index{packages!#1@\texttt{#1}}% package name
    }%
}% command name -- adds backslash automatically
\newcommand{\docopt}[1]{\ensuremath{\langle}\textrm{\textit{#1}}\ensuremath{\rangle}}% optional command argument
\newcommand{\docarg}[1]{\textrm{\textit{#1}}}% (required) command argument
\newenvironment{docspec}{\begin{quotation}\ttfamily\parskip0pt\parindent0pt\ignorespaces}{\end{quotation}}% command specification environment
\newcommand{\docenv}[1]{\texttt{#1}\index{#1 environment@\texttt{#1} environment}\index{environments!#1@\texttt{#1}}}% environment name
\newcommand{\docenvdef}[1]{\hlred{\texttt{#1}}\label{env:#1}\index{#1 environment@\texttt{#1} environment}\index{environments!#1@\texttt{#1}}}% environment name
\newcommand{\docpkg}[1]{\texttt{#1}\index{#1 package@\texttt{#1} package}\index{packages!#1@\texttt{#1}}}% package name
\newcommand{\doccls}[1]{\texttt{#1}}% document class name
\newcommand{\docclsopt}[1]{\texttt{#1}\index{#1 class option@\texttt{#1} class option}\index{class options!#1@\texttt{#1}}}% document class option name
\newcommand{\docclsoptdef}[1]{\hlred{\texttt{#1}}\label{clsopt:#1}\index{#1 class option@\texttt{#1} class option}\index{class options!#1@\texttt{#1}}}% document class option name defined
\newcommand{\docmsg}[2]{\bigskip\begin{fullwidth}\noindent\ttfamily#1\end{fullwidth}\medskip\par\noindent#2}
\newcommand{\docfilehook}[2]{\texttt{#1}\index{file hooks!#2}\index{#1@\texttt{#1}}}
\newcommand{\doccounter}[1]{\texttt{#1}\index{#1 counter@\texttt{#1} counter}}

\newcommand{\studyq}[1]{\marginnote{Q: #1}}

\hypersetup{colorlinks}% uncomment this line if you prefer colored hyperlinks (e.g., for onscreen viewing)

% Generates the index
\usepackage{makeidx}
\makeindex

\setcounter{tocdepth}{3}
\setcounter{secnumdepth}{3}

%%%%%%%%%%%%%%%%%%%%%%%%%%%%%%%%%%%%%%%%%%%%%%%%%%%%%%%%%%%%%%
% custom commands

\newtheorem{theorem}{\color{pastel-blue}Theorem}[section]
\newtheorem{lemma}[theorem]{\color{pastel-blue}Lemma}
\newtheorem{proposition}[theorem]{\color{pastel-blue}Proposition}
\newtheorem{corollary}[theorem]{\color{pastel-blue}Corollary}

\newenvironment{proof}[1][Proof]{\begin{trivlist}
\item[\hskip \labelsep {\bfseries #1}]}{\end{trivlist}}
\newenvironment{definition}[1][Definition]{\begin{trivlist}
\item[\hskip \labelsep {\bfseries #1}]}{\end{trivlist}}
\newenvironment{example}[1][Example]{\begin{trivlist}
\item[\hskip \labelsep {\bfseries #1}]}{\end{trivlist}}
\newenvironment{remark}[1][Remark]{\begin{trivlist}
\item[\hskip \labelsep {\bfseries #1}]}{\end{trivlist}}

\hyphenpenalty=5000

% more pastel ones
\xdefinecolor{pastel-red}{rgb}{0.77,0.31,0.32}
\xdefinecolor{pastel-green}{rgb}{0.33,0.66,0.41}
\definecolor{pastel-blue}{rgb}{0.30,0.45,0.69} % crayola blue
\definecolor{gray}{rgb}{0.2,0.2,0.2} % dark gray

\xdefinecolor{orange}{rgb}{1,0.45,0}
\xdefinecolor{green}{rgb}{0,0.35,0}
\definecolor{blue}{rgb}{0.12,0.46,0.99} % crayola blue
\definecolor{gray}{rgb}{0.2,0.2,0.2} % dark gray

\xdefinecolor{cerulean}{rgb}{0.01,0.48,0.65}
\xdefinecolor{ust-blue}{rgb}{0,0.20,0.47}
\xdefinecolor{ust-mustard}{rgb}{0.67,0.52,0.13}

%\newcommand\comment[1]{{\color{red}#1}}

\newcommand{\dy}{\partial}
\newcommand{\ddy}[2]{\frac{\dy#1}{\dy#2}}

\newcommand{\ab}{\boldsymbol{a}}
\newcommand{\bb}{\boldsymbol{b}}
\newcommand{\cb}{\boldsymbol{c}}
\newcommand{\db}{\boldsymbol{d}}
\newcommand{\eb}{\boldsymbol{e}}
\newcommand{\lb}{\boldsymbol{l}}
\newcommand{\nb}{\boldsymbol{n}}
\newcommand{\tb}{\boldsymbol{t}}
\newcommand{\ub}{\boldsymbol{u}}
\newcommand{\vb}{\boldsymbol{v}}
\newcommand{\xb}{\boldsymbol{x}}
\newcommand{\wb}{\boldsymbol{w}}
\newcommand{\yb}{\boldsymbol{y}}

\newcommand{\Xb}{\boldsymbol{X}}

\newcommand{\ex}{\mathrm{e}}
\newcommand{\zi}{{\rm i}}

\newcommand\Real{\mbox{Re}} % cf plain TeX's \Re and Reynolds number
\newcommand\Imag{\mbox{Im}} % cf plain TeX's \Im

\newcommand{\zbar}{{\overline{z}}}

\newcommand\Def[1]{\textbf{#1}}

\newcommand{\qed}{\hfill$\blacksquare$}
\newcommand{\qedwhite}{\hfill \ensuremath{\Box}}

%%%%%%%%%%%%%%%%%%%%%%%%%%%%%%%%%%%%%%%%%%%%%%%%%%%%%%%%%%%%%%
% some extra formatting (hacked from Patrick Farrell's notes)
%  https://courses.maths.ox.ac.uk/node/view_material/4915
%

% chapter format
\titleformat{\chapter}%
  {\huge\rmfamily\itshape\color{pastel-red}}% format applied to label+text
  {\llap{\colorbox{pastel-red}{\parbox{1.5cm}{\hfill\itshape\huge\color{white}\thechapter}}}}% label
  {1em}% horizontal separation between label and title body
  {}% before the title body
  []% after the title body

% section format
\titleformat{\section}%
  {\normalfont\Large\itshape\color{pastel-green}}% format applied to label+text
  {\llap{\colorbox{pastel-green}{\parbox{1.5cm}{\hfill\color{white}\thesection}}}}% label
  {1em}% horizontal separation between label and title body
  {}% before the title body
  []% after the title body

% subsection format
\titleformat{\subsection}%
  {\normalfont\large\itshape\color{pastel-blue}}% format applied to label+text
  {\llap{\colorbox{pastel-blue}{\parbox{1.5cm}{\hfill\color{white}\thesubsection}}}}% label
  {1em}% horizontal separation between label and title body
  {}% before the title body
  []% after the title body

%%%%%%%%%%%%%%%%%%%%%%%%%%%%%%%%%%%%%%%%%%%%%%%%%%%%%%%%%%%%%%%%%%%%%%%%%%%%%%%%

\begin{document}

% Front matter
%\frontmatter

% r.3 full title page
%\maketitle

% v.4 copyright page

\chapter*{}

\begin{fullwidth}

\par \begin{center}{\Huge Algebra 1H}\end{center}

\vspace*{5mm}

\par \begin{center}{\Large typed up by B. S. H. Mithrandir}\end{center}

\vspace*{5mm}

\begin{itemize}
  \item \textit{Last compiled: \monthyear}
  \item Adapted from notes of N. Martin, Durham
  \item This was part of the Durham Core A module given in the first year. This
  is an introduction to group theory, number theory, and proofs.
  \item[]
  \item \TODO diagrams
\end{itemize}

\par

\par Licensed under the Apache License, Version 2.0 (the ``License''); you may not
use this file except in compliance with the License. You may obtain a copy
of the License at \url{http://www.apache.org/licenses/LICENSE-2.0}. Unless
required by applicable law or agreed to in writing, software distributed
under the License is distributed on an \smallcaps{``AS IS'' BASIS, WITHOUT
WARRANTIES OR CONDITIONS OF ANY KIND}, either express or implied. See the
License for the specific language governing permissions and limitations
under the License.
\end{fullwidth}


%===============================================================================

\chapter{Groups and numbers}

A \Def{group} $G$ is a non-empty set with structure coming from a \Def{binary
group operatation}, usually denotes $\circ$. For every pair $g_1, g_2 \in G$,
there exists $g_1 \circ g_2$ ($g_1$ is \Def{composed} with $g_2$). To be a
group, the following conditions needs to be satisfied:
\begin{enumerate}
	\item \Def{Closure}: for all $g_1, g_2 \in G$, $g_1 \circ g_2 \in G$.
	\item \Def{Associativity}: for all $g_1, g_2, g_3 \in G$, $(g_1 \circ
	g_2) \circ g_3 = g_1 \circ (g_2 \circ g_3)$.
	\item \Def{Identity}: There exists $e\in G$ such that, for all $g\in G$,
	$e\circ g = g\circ e = g$.
	\item \Def{Inverse}: There exists $g_i^{-1}\in G$ such that, for all
	$g_i \in G$, $g_i^{-1} \circ g_i = g_i \circ g_i^{-1}=e$, and $g_i^{-1}\neq
	g_j^{-1}$ if $i\neq j$.
\end{enumerate}

\begin{example}

\begin{enumerate}
	\item Let $\mathsf{M}(m,n,\mathbb{R})$ be an $m\times n$ matrix with real
	coefficients. Under matrix multiplication, it is a group is $m=n$ and
	$|\mathbb{M}|\neq0$; this is called the \Def{general linear group}
	$\mbox{GL}(n,\mathbb{R})$. Note this group is not commutative (the ordering of
	composition matters).
	
	\item $C_n=\{\exp(2k\pi\zi/n)\ |\ 0\leq k\leq n-1\}$ is a group under
	multiplication: $1=\ex^{2\pi \zi}$ is the identity element, and it is closed
	if we remove the excess multiples of $\ex^{2\pi \zi}$. With this, the inverse
	is easily defined, and it is associative by properties of multiplication. This
	is the \Def{cyclic} group of $n$ elements, with $\exp(2\pi\zi/n)$ being
	the \Def{generator} (more later).
	
	\item $G=\{-1,1\}$ under multiplication and
	$H=\{\textnormal{even},\textnormal{odd}\}$ under addition of numbers are both
	groups. In particular, there is a one-to-one identification between
	$1\leftrightarrow\textnormal{even}$ and $-1\leftrightarrow\textnormal{odd}$,
	so the two groups have similar structure. $G$ is actually
	\Def{isomorphic} to $H$, denoted $G\cong H$.
	
	\item A non-square rectangle has the symmetries $\{I,H,V,R\}$ which are,
	respectively, the identity (i.e., doing nothing), horizontal reflection,
	vertical reflection, and rotation by $\pi$. A group table may be formed (row
	first, then column). Contrast this with $C_4=\{1,-1,\zi,-\zi\}$ under
	multiplication:
	\begin{table}[h]
	\begin{center}\begin{tabular}{c|cccc}
		$\circ$ & $I$ & $H$ & $V$ & $R$\\
		\hline
		$I$ & $I$ & $H$ & $V$ & $R$\\
		$H$ & $H$ & $I$ & $R$ & $V$\\
		$V$ & $V$ & $R$ & $I$ & $H$\\
		$R$ & $R$ & $V$ & $H$ & $I$
	\end{tabular}\qquad\qquad\begin{tabular}{c|cccc}
		$\times$ & $1$ & $-1$ & $\zi$ & $-\zi$\\
		\hline
		$1$ & $1$ & $-1$ & $\zi$ & $-\zi$\\
		$-1$ & $-1$ & $1$ & $-\zi$ & $\zi$\\
		$\zi$ & $\zi$ & $-\zi$ & $-1$ & $1$\\
		$-\zi$ & $-\zi$ & $\zi$ & $1$ & $-1$
	\end{tabular}\end{center}
	\end{table}
	The two have different structures, so are not isomorphic.
\end{enumerate}
\end{example}

\begin{lemma}
	The identity and the inverse in a group is unique.
\end{lemma}
\begin{proof}
	Suppose $e,f\in G$ are identity elements, then
	\begin{equation*}
		ef=e,\qquad ef=f\qquad\Rightarrow\qquad e=f,
	\end{equation*}
	so we have uniqueness. Suppose $h$ and $k$ are both inverses to $g$, then
	\begin{equation*}
		h=he=h(gk)=(hg)k=ek=k,
	\end{equation*}
	so we also have uniqueness. \qed
\end{proof}

%===============================================================================

\section{Numbers}

\begin{theorem}
	Let $n,m\in\mathbb{Z}$, $m>0$. There exists $q,r\in\mathbb{Z}$ such that
	$n=qm+r$, $0\leq r<n$. (Here, $q$ is quotient, $r$ is remainder.) \qedwhite
\end{theorem}
We say $m$ divides $n$ ($m|n$) is there exists $q$ such that $n=mq$, i.e.,
$r=0$.
\begin{lemma}
	\begin{enumerate}
		\item For all $n$, $n|0$.
		\item For all $n$, $n|1$.
		\item For all $n$, $n|n$.
		\item $l|m$ and $m|n$ implies $l|n$.
		\item If $n\neq0$, $0\nmid n$.
		\item $n|a$ and $n|b$ implies that $n|(a\pm b)$. \qedwhite
	\end{enumerate}
\end{lemma}

Prime numbers have exactly two distinct factors (so $1$ is not prime).
\begin{lemma}
	If $n$ is not prime, there exists a prime $p\leq\sqrt{n}$ such that $p|n$.
\end{lemma}
\begin{proof}
	If $n$ is not prime, then there are at least three factors, and every such
	divisor is less than or equation to $n$. Let $p>1$ be the smallest divisor of
	$n$. $p$ is prime because if there is a $k$ where $k|p$, then $k|p|n$ and $p$
	is not the smallest divisor of $n$. $p|n$ so $n=pq$, thus $p\leq\sqrt{n}$,
	otherwise $q$ would be a smaller non-trivial factor of $n$. \qed
\end{proof}

\begin{theorem}[Fundamental theorem of arithmetic]
	Let $n\in\mathbb{Z}$, $|n|>1$. It is possible to write
	\begin{equation*}
	  n=\pm p_1^{r_1}p_2^{r_2}\cdots p_k^{r_k},
	\end{equation*}
	where $k\geq1$, $p_1<p_2<\cdots<p_k$ are prime numbers, and, for all
	$i\in\mathbb{N}$, $r_i\geq1$, i.e., all integers may be written as a product
	of prime numbers. \qedwhite
\end{theorem}
\begin{theorem}
  There are infinite many prime numbers.
\end{theorem}
\begin{proof}
  We carry out a proof by contradiction. Suppose there are finite number of
  primes with $0 < p_1 < p_2 < \cdots < p_k$. Let $n=p_1 p_2 \cdots p_k+1$. For
  all $i$, diving by $p_i$ gives remainder $1$. However, the fundamental theorem
  of arithmetic guarantees $n$ may be factorised as primes, therefore the list
  above is not complete. \qed
\end{proof}

%-------------------------------------------------------------------------------

\subsection{Common factors}

\begin{example}
  The numbers $336$ and $231$ have the greatest common divisor (gcd) of $21$:
  \begin{equation*}
    336 = 231+105,\qquad 231 = 2\times105+21,\qquad 105=5\times21+0.
  \end{equation*}
  We write $\mbox{gcd}(336,231)=21$.
\end{example}
Here, we can define an algorithm that generates the $\mbox{gcd}$ of any two
integers.
\begin{proposition}[Euclidean algorithm]
  Given $m,n\in\mathbb{Z}^+$, the following algorithm generates
  $\mbox{gcd(m,n)}$:
  \begin{enumerate}
    \item If $m>n$, swap so $n>m$;
    \item $n=q\cdot m+r$, $0\leq r<m$;
    \item If $r=0$, output $m$ as $\mbox{gcd}$ and stop;
    \item Otherwise, replace $n=m$, $m=r$, and repeat from step 2. \qedwhite
  \end{enumerate}
\end{proposition}

\begin{theorem}
  Euclidean algorithm generates $\mbox{gcd}(m,n)$.
\end{theorem}
\begin{proof}
  Let $d$ be any common divisor, then $d|m$ and $d|n$, and thus $d|(n-qm)=r$. At
  each stage the same divisor divides each $m$, $n$ and $r$ until $r=0$, and
  current value of $m$ is our output number, the $\mbox{gcd}$. \qed
\end{proof}

\begin{corollary}
  Given any $m,n\in\mathbb{Z}^+$ with $\mbox{gcd}(m,n)=d$, we can always write
  $d=mx+ny$ for $x,y\in\mathbb{Z}$.
\end{corollary}
\begin{proof}
  At each step of the Euclidean algorithm, we can write the current values of
  $m$, $n$ and $r$ as a linear combination of the original values. The
  $k^{\textnormal{th}}$ step of the iteration makes us solve
  \begin{equation*}
    n_k = q_k m_k + r_k\qquad\Leftrightarrow r_k = n_k - q_k m_k.
  \end{equation*}
  For all $k$, $r_k$ is a linear combination in the cycle of iterations. Process
  starts with original $m$, $n$ and ends with the $\mbox{gcd}$ in the form of a
  linear combination. \qed
\end{proof}
\begin{example}
  $21=\mbox{gcd}(336,231)$. We have
  \begin{itemize}
    \item $336=231+105$, $105=336-231$.
    \item $231=2\times105+21$, $21=231-2(336-331)=3\times231-2\times336$.
  \end{itemize}
\end{example}

%-------------------------------------------------------------------------------

\subsection{Modular arithmetic}

\begin{theorem}
  There are infinitely many primes of the form $4k+3$.
\end{theorem}
\begin{proof}
  Suppose this is false, then there is a largest prime $n$, $n\geq3$. Let
  $N=(4n)!-1=4m-1$, $m\in\mathbb{Z}$. By the fundamental theorem of arithmetic,
  since $(4m-1)$ is odd, we see all primes involved are odd. Everyone of our
  original list of primes of the form $4k+3$ gives remainder $-1$ when divided
  into $N$, so none of these are factors.
  
  All factors of $N$ thus have the form $4\ell+1$, $\ell\in\mathbb{Z}$. All
  products of $4\ell+1$ results in a number $4\ell'+1$ which is a contraction to
  the statement that prime products have the form $4m-1$. \qed
\end{proof}

For $n\in\mathbb{Z}$, $a,b\in\mathbb{Z}$ are \Def{congruent modulo $n$} if
$n|(a-b)$, denoted $a\equiv b\ (\mbox{mod}\ n)$. Since $a\equiv b\ (\mbox{mod}\
n)$ iff $a=b+nk$ for $k\in\mathbb{Z}$. We see this may also define an
equivalence relation.
\begin{example}
  \begin{align*}
    27\equiv 2\ (\mbox{mod}\ 5),\quad 101\equiv 24\ (\mbox{mod}\ 11),\\
    -37\equiv 53\ (\mbox{mod}\ 1),\quad 10^n-1\equiv 0\ (\mbox{mod}\ 9).
  \end{align*}
\end{example}

The \Def{congruence class of $a$ mod $n$} is defined to be
$\overline{a}=\{a+kn\ |\ k\in\mathbb{Z}\}$.
\begin{example}
  The congruence class of $0$ mod $5$ and $1$ mod $5$ are respectively
  \begin{equation*}
    \overline{0}=\{\cdots,-5,0,5,\cdots\},\qquad
    \overline{1}=\{\cdots,-4,1,4,\cdots\}.
  \end{equation*}
\end{example}
There are only five distinct congruence classes in mod $5$, represented by the
\Def{principal residues} in the range $\overline{0},\cdots\overline{4}$.
In general, for $n\in\mathbb{Z}$, there are $n$ distinct congruence classes mod
$n$, represented by $\overline{0},\overline{1},\cdots\overline{n-1}$.

For general $n$, we take the set of integer mod $n$ as $\mathbb{Z}_n$ (or
$\mathbb{Z}_n/\mathbb{Z}$). We can sometimes solve $ax\equiv b\ (\mbox{mod}n)$
for $x$. For example, $7x\equiv14$ (mod $35$) may be reduced to $x\equiv2$ (mod
$5$), and so $x=2+5n\in\mathbb{Z}_{35}$. However, we see $7x\equiv15$ (mod $35$) cannot be solved for $x\in\mathbb{Z}$ since $7\nmid15$, but $7|14$ and $7|35$.

\begin{proposition}
  $\mathbb{Z}_n$ is a group under addition. $\overline{0}$ acts like zero, we
  have closure, associativity from addition, and the inverse of $\overline{a}$
  is given by $\overline{n-a}$. \qedwhite
\end{proposition}
In addition, $\mathbb{Z}_n$ is a cyclic group with generator $\overline{1}$.

\begin{proposition}
  Let $p$ be prime, $\overline{a}\neq\overline{0}\in\mathbb{Z}_p$, then:
  \begin{itemize}
    \item there exists $\overline{b}$ such that
    $\overline{a}\times\overline{b}=\overline{1}$;
    \item for all $\overline{c}\in\mathbb{Z}_p$, there exists $\overline{x}$
    such that $\overline{a}\times\overline{x}=\overline{c}$;
    \item $\mathbb{Z}_p-\{\overline{0}\}$ is a group under multiplication.
  \end{itemize}
\end{proposition}
\begin{proof}
  \begin{itemize}
    \item If $p$ is prime and $a\neq0$ (mod $p$), then $\mbox{gcd}(a,p)=1$. So
    there exists $b$ and $c$ such that $ab+pc=1$, but
    \begin{equation*}
      1=ab+pc\equiv ab\ (\mbox{mod}\ p),
    \end{equation*}
    so $\overline{a}\times\overline{b}=\overline{1}$ in $\mathbb{Z}_p$ with
    $b\neq0$.
    \item From the previous part, $\overline{c}=\overline{c}\overline{1}=
    \overline{c}(\overline{a}\overline{b})=
    \overline{a}(\overline{c}\overline{b})$. Let
    $\overline{x}=\overline{c}\overline{b}$, and we have the result.
    \item Associativity is trivial. $\overline{1}$ is the identity, and we
    proved existence of the inverse in the previous parts. \qed
  \end{itemize}
\end{proof}

\begin{lemma}
  Let $0<a<n$, $a,n\in\mathbb{Z}$, $\mbox{gcd}(a,n)=1$. Then there exists $b$
  with $0<b<n$ such that $ab\equiv1\ (\mbox{mod}\ n)$.
\end{lemma}
\begin{proof}
  $\mbox{gcd}(a,n)=1$ implies that we have $ax+ny=1$ for some
  $x,y\in\mathbb{Z}$. Select a $b$ such that $b\equiv x\ (\mbox{mod}\ n)$
  implies that $ab\equiv ax=1-ny\equiv1\ (\mbox{mod}\ n)$.
  
  Suppose $b$ is not unique, and $b'$ also exists. Working in mod $n$,
  \begin{equation*}
    \overline{b'}=\overline{b'}\cdot\overline{1}
    =\overline{b'}(\overline{a}\overline{b})
    =(\overline{b'}\overline{a})\overline{b}=\overline{1}\cdot\overline{b}
    =\overline{b},
  \end{equation*}
  so $\overline{b}$ is unique. \qed
\end{proof}

Two numbers $a$ and $b$ are \Def{co-prime} if $\mbox{gcd}(a,b)=1$.

$\mathbb{Z}_n-\{0\}$ is not generally a group under multiplication. Let
$n\geq2$, $n\in\mathbb{Z}$, then we define
\begin{equation*}
  \mathbb{Z}_n^* = \{\overline{r}\ |\ 1\leq r\leq n,\ \mbox{gcd}(r,n)=1\}.
\end{equation*}
We observe that $\mathbb{Z}_n^* \subseteq\mathbb{Z}_n$. We have, for example,
\begin{equation*}
  \mathbb{Z}_3^*=\{\overline{1},\overline{2}\},\qquad
  \mathbb{Z}_4^*=\{\overline{1},\overline{3}\},\qquad
  \mathbb{Z}_9^*=\{\overline{1},\overline{2},\overline{4},\overline{5},
  \overline{7},\overline{8}\}.
\end{equation*}

\begin{proposition}
  We have that:
  \begin{enumerate}
    \item $\mathbb{Z}_n^*$ is well defined;
    \item $\mathbb{Z}_n^*$ is closed under multiplication;
    \item the inverse of a residue is also in $\mathbb{Z}_n^*$;
    \item if $\overline{a},\overline{b},\overline{c}\in\mathbb{Z}_n^*$, with
    $\overline{a}\overline{b}=\overline{a}\overline{c}$, then
    $\overline{b}=\overline{c}$;
    \item $\mathbb{Z}_n^*$ is a group under multiplication.
  \end{enumerate}
\end{proposition}
\begin{proof}
  In order:
  \begin{enumerate}
    \item We see a residue is represented by all others that are congruent to
    it. However, if $k$ is a number, $k\equiv x$ (mod $n$), then $x=k+tn$,
    $t\in\mathbb{Z}$. So $\mbox{gcd}(k,n)=1$ iff $\mbox{gcd}(x,n)=1$, so
    $\mathbb{Z}_n^*$ is well defined.
    \item $\mbox{gcd}(a,n)=1$ and $\mbox{gcd}(b,n)=1$ implies that
    $\mbox{gcd}(ab,n)=1$, so we have closure.
    \item There exists $x$ and $y$ where $ax+ny=1$, so $\mbox{gcd}(ax,n)=1$,
    which implies $\mbox{gcd}(x,n)=1$, and so inverse exists and belongs to
    $\mathbb{Z}_n^*$ from the previous point.
    \item Let $x$ be the inverse residue, then
    \begin{equation*}
      \overline{x}\overline{a}\overline{b}=\overline{x}\overline{a}\overline{c}
      \qquad\Rightarrow\qquad
      \overline{1}\overline{b}=\overline{1}\overline{c},
    \end{equation*}
    and $\overline{b}=\overline{c}$.
    \item We have proved this from the above points. \qed
  \end{enumerate}
\end{proof}

\begin{theorem}
  In modulo $n$, $n\geq2$, let $a>0$, $c\geq0$, $a,c\in\mathbb{Z}$:
  \begin{enumerate}
    \item if $\mbox{gcd}(a,n)=1$, there exists $x$ with $0\leq x<n$ where
    $ax\equiv c\ (\mbox{mod}\ n)$, and $x$ is unique;
    \item if $\mbox{gcd}(a,n)=d>1$ and $d\nmid c$, then there is no $x$ where
    $ax\equiv c\ (\mbox{mod}\ n)$;
    \item if $\mbox{gcd}(a,n)=d>1$ and $d|c$, there are $d$ values of $x$,
    $0\leq x<n$ such that $ax\equiv c\ (\mbox{mod}\ n)$.
  \end{enumerate}
\end{theorem}
\begin{proof}
  As follows:
  \begin{enumerate}
    \item If $\mbox{gcd}(a,n)=1$ and, then $\overline{a}\in\mathbb{Z}_n^*$, so
    there exists $\overline{y}\in\mathbb{Z}_n^*$ such that
    $\overline{a}\overline{y}=1$. Let $x$ be the residue of $yc$ in mod $n$,
    then we get
    \begin{equation*}
      ax=(ay)c\equiv1\cdot c=c\ (\mbox{mod}\ n),
    \end{equation*}
    so $x$ exists. Suppose $x'$ is another such residue, then $ax-ax'\equiv0\
    (\mbox{mod}\ n)$, and so
    \begin{equation*}
      x-x'\equiv1\cdot(x-x')\equiv ya(x-x')=y(ax-ax')\equiv0\ (\mbox{mod}\ n).
    \end{equation*}
    \item $ax\equiv c\ (\mbox{mod}\ n)$ implies that $ax=c+kn$ for some $k$.
    Thus $c=ax-kn$, and $\mbox{gcd}(a,n)=d$ necessarily implies that $d|c$, so
    we have a contradiction.
    \item Here, there exists $b,e,m\in\mathbb{Z}$ such that $a=bd$, $c=ed$ and
    $n=md$. We have $\mbox{gcd}(b,m)=1$, and so by (i) there exists an unique
    $t$ with $0\leq t<m$ such that $bt=e\ (\mbox{mod}\ m)$. The claim is that
    $x=t+rm$ with $0\leq r\leq d-1$ are the $d$ solutions to the original
    equation $ax\equiv\ (\mbox{mod}\ n)$. This is because
    \begin{equation*}
      a(t+rm)=bd(t+rm)=d(bt)+br(dm),
    \end{equation*}
    and since $bt\equiv c\ (\mbox{mod}\ m)$, this implies that
    \begin{equation*}
      d(bt)+br(dm)=d(e+km)+brn=de+k(dm)+brn=c+kn+brn=c+(k+br)n.
    \end{equation*}
    Indeed, $x+t+rm$ are the solutions to $ax\equiv c\ (\mbox{mod}\ n)$.
    
    If $x$ and $x'$ are distinct solutions, then $a(x-x')\equiv0\ (\mbox{mod}\
    n)$, and $a(x-x')=kn$. This means that we have $db(x-x')=kdm$, thus
    $b(x-x')=km$, and so $b(x-x')=0\ (\mbox{mod}\ n)$. Hence
    \begin{equation*}
      \mbox{gcd}(b,m)=1\qquad\Rightarrow\qquad x-x'\equiv0\ (\mbox{mod}\ m)
    \end{equation*}
    as required. \qed
  \end{enumerate}
\end{proof}

\begin{example}
  \begin{enumerate}
    \item $9x\equiv8\ (\mbox{mod}\ 23)$. We have $\mbox{gcd}(9,23)=1$, and we
    see that $1=2\cdot23-5\cdot9$, so $-5\cdot9\equiv1\ (\mbox{mod}\ 23)$; thus
    \begin{equation*}
      x\equiv(-5\cdot9)x\equiv-5\cdot(9x)\equiv-5\cdot8\equiv-40\equiv6\
      (\mbox{mod}\ 23).
    \end{equation*}
    \item $10x\equiv14\ (\mbox{mod}\ 18)$. Now, $\mbox{gcd}(10,18)=2$, and we
    see that $10x=14+18k$ is equivalent to $5x=7+9k$, and now we have
    $5x\equiv7\ (\mbox{mod}\ 9)$ and $\mbox{gcd}(5,7)=1$. Since
    $1=2\cdot5-1\cdot9$, $2\cdot5\equiv1\ (\mbox{mod}\ 9)$, and
    \begin{equation*}
      x\equiv(2\cdot5)x\equiv2\cdot7\equiv14\equiv5\ (\mbox{mod}\ 9).
    \end{equation*}
    By the theorem, there should be two distinct values of $x$, and so $x=5,14$.
    \item $25x\equiv65\ (\mbox{mod}\ 90)$. Here, $\mbox{gcd}(25,90)=5$, and
    diving through by $5$ gives $5x=13+18k$, and now $5x\equiv13\ (\mbox{mod}\
    18)$, $\mbox{gcd}(5,13)=1$, with $1=2\cdot18-7\cdot5$. Thus
    \begin{equation*}
      x\equiv(-7\cdot5)x\equiv-7\cdot13\equiv-91\equiv17\ (\mbox{mod}\ 18),
    \end{equation*}
    with $x=17,35,53,71,89$.
    \item $20x\equiv65\ (\mbox{mod}\ 90)$. Here, $\mbox{gcd}(20,10)=10$,
    however, $10\nmid65$, so there are no solutions in $\mathbb{Z}$.
  \end{enumerate}
\end{example}
\begin{corollary}[Chinese remainder theorem]
  Suppose $\mbox{gcd}(m,n)=1$, $0\leq a<m$ and $0\leq b<n$. Then there exists an
  unique $c$ with $0\leq c<mn$ such that $c\equiv a\ (\mbox{mod}\ m)$ and
  $c\equiv b\ (\mbox{mod}\ n)$.
\end{corollary}
\begin{proof}
  We need $c=a+km$ and $c=b+ln$. Thus $km=c-a\equiv b-a\ (\mbox{mod}\ n)$. Now,
  $\mbox{gcd}(m,n)=1$, so there exists $x$ and $y$ such that $mx+ny=1$. Choosing
  $c=a+x(b-a)m$ gives $c\equiv a\ (\mbox{mod}\ m)$. Now, $mx=1-ny$ gives
  \begin{equation*}
    c=a+(b-a)(1-my)=b+y(a-b)n,
  \end{equation*}
  so $c\equiv b\ (\mbox{mod}\ n)$ also. \qed
\end{proof}

\begin{example}
  With $c\equiv 6\ (\mbox{mod}\ 8)$ and $c\equiv13\ (\mbox{mod}\ 15)$, we have
  $0\leq c<8\cdot15=120$, and noticing $2\cdot8-1\cdot15=1$, we have $x=2$, and
  $c=6+2(13-6)8=118$.
\end{example}

%-------------------------------------------------------------------------------

\subsection{Totient function}

Let the number of elements in $\mathbb{Z}_n^*$ be denoted by $\phi(n)$, the
\Def{Euler $\phi$-function}, also called the \Def{totient function}.
For $n\geq3$, $\phi(n)$ is always even, while for $p$ prime, $\phi(p)=p-1$, and
$\phi(p^n)=p^n-p^{n-1}$. If $\mbox{gcd}(m,n)=1$, then $\phi(mn)=\phi(m)\phi(n)$.

\begin{theorem}[Euler--Fermat theorem]
  Let $n>1$, $\mbox{gcd}(a,n)=1$. Then $a^{\phi(n)}\equiv1\ (\mbox{mod}\ n)$.
\end{theorem}
\begin{proof}
  Let $[\overline{x_1},\overline{x_2},\cdots,\overline{x_{\phi(n)}}]$ be a list
  of all distinct elements of $\mathbb{Z}_n^*$. Let
  $z=\prod_{i=1}^{\phi(n)}\overline{x_i}$. Now consider
  $[a\overline{x_1},a\overline{x_2},\cdots,a\overline{x_{\phi(n)}}]$. By
  proposition, all elements are distinct, and all elements are in
  $\mathbb{Z}_n^*$, i.e., the list is a permutation of the original list. Thus
  \begin{equation*}
    \overline{z}=\prod_{i=1}^{\phi(n)}(a\overline{x_i})=a^{\phi(n)}\overline{z},
  \end{equation*}
  so $1\equiv a^{\phi(n)}\ (\mbox{mod}\ n)$. \qed
\end{proof}

\begin{example}[Example: public key cryptography]
  This above idea is used in public key cryptography. The idea is that Alice
  sends Bob a secure message $T$. Bob has a public method of encoding the
  message (the \Def{public key}). Alice encodes $T$ to $M$ and sends this to
  Bob. Bob has a secret way to decode $M$ to recover $T$.
  
  Bob chooses two very large and distinct prime numbers $p$ and $q$. He also
  chooses two very large numbers $d$ and $e$ such that
  \begin{equation*}
    de\equiv1\ (\mbox{mod}\ (p-1)q-1)).
  \end{equation*}
  Bob makes $e$ public.
  
  Alice converts her message into numbers all less than $p$ and $q$. Let $T$ be
  one such number. Alice works out the residue $M\equiv T^e\ (\mbox{mod}\ pq)$
  and sends $M$. Bob works out the residue $U\equiv M^d=(T^e)^d$, and $U=T$. To
  show this, we observe that, since $T<q$ and $T<p$, $\mbox{gcd}(T,pq)=1$. By
  the Euler--Fermat theorem, $T^{\phi(pq)}\equiv1\ (\mbox{mod}\ pq)$. Since $p$
  and $q$ are co-prime,
  \begin{equation*}
    \phi(pq)=\phi(p)\phi(q)=(p-1)(q-1).
  \end{equation*}
  Bob chooses $ed\equiv 1\ (\mbox{mod}\ (p-1)(q-1)=\phi(pq))$, so
  \begin{equation*}
    ed=k\phi(pq)+1,\qquad k\in\mathbb{Z}.
  \end{equation*}
  Thus
  \begin{equation*}
    (T^e)^d=T^{k\phi(pq)+1}=T^{k\phi(pq)}T=[T^\phi(pq)]^k T\equiv
    1^k\cdot T=T\ (\mbox{mod}\ pq).
  \end{equation*}
  
  As an example, consider $p=7$, $q=13$. Then $pq=91$, and
  $\phi(pq)=(7-1)(13-1)=72$. We need $e$ and $d$ to be co-prime to $72$, and
  mutually inverse in $\mathbb{Z}_{72}^*$; we observe that $e=5$ and $d=79$
  works. Suppose $T=10$ is the thing we are sending; observe that
  $\mbox{gcd}(10,7)=\mbox{gcd}(10,13)=1$.
  
  To encode, we have $T^e=10^5=1098\cdot91+82\equiv82\ (\mbox{mod}\ 91)$. To
  decode, $82^d=82^29\equiv10\ (\mbox{mod}\ 91)$, as required.
\end{example}

Two groups $G$ and $H$ are \Def{isomorphic}, $G\cong H$ if there is a
mapping
$\alpha:G\rightarrow H$ such that:
\begin{enumerate}
  \item $\alpha$ is a \Def{homomorphism}, i.e., $\alpha(g_1 \circ
  g_2)=\alpha(g_1)\circ\alpha(g_2)$;
  \item $\alpha$ is bijective, i.e., injective and surjective.
\end{enumerate}

If $G$ and $H$ are two groups, then the \Def{Cartesian product} is defined
to be
\begin{equation*}
  G\times H=\{(g,h)\ |\ g\in G,\ h\in H\},\qquad
  (g_1,h_1)\circ(g_2,h_2)=(g_1 \circ g_2, h_1 \circ h_2).
\end{equation*}
With this, the identity element in $G\times H$ is $(e_G, e_H)$, the inverse is
$(g,h)^{-1}=(g^{-1}, h^{-1})$.
\begin{example}
  For $\mathbb{Z}_m\times\mathbb{Z}_n$, with addition being the operation we
  have:
  \begin{enumerate}
    \item closure with $(\overline{a_1},\overline{b_1})+
    (\overline{a_2},\overline{b_2})=(\overline{a_1+a_2},\overline{b_1+b_2})$;
    \item associativity by inheritance;
    \item identity is $(\overline{0},\overline{0})$;
    \item the inverse to $(\overline{a},\overline{b})$ is
    $(-\overline{a},-\overline{b})$.
  \end{enumerate}
  So $\mathbb{Z}_m\times\mathbb{Z}_n$ is a group under addition, with
  $|\mathbb{Z}_m\times\mathbb{Z}_n|=mn$.
\end{example}

\begin{theorem}
  If $m$ and $n$ are co-prime, then $\mathbb{Z}_m\times\mathbb{Z}_n
  \cong\mathbb{Z}_{mn}$.
\end{theorem}
\begin{proof}
  Observe that $(\overline{1},\overline{1})\in\mathbb{Z}_m\times\mathbb{Z}_n$ is
  the identity, corresponding to $\overline{1}\in\mathbb{Z}_{mn}$. We define
  \begin{equation*}
    \phi:\mathbb{Z}_{mn}\rightarrow\mathbb{Z}_m\times\mathbb{Z}_n,\qquad
    \phi(\overline{k})=k(\overline{1},\overline{1})=(\overline{k},\overline{k}).
  \end{equation*}
  (We will be calculating in the correct modulos are required.) Suppose
  $\phi(\overline{k})=\phi(\overline{l})$, then $k\equiv l\ (\mbox{mod}\ m)$ and
  $k\equiv l\ (\mbox{mod}\ n)$. Thus $m|(k-l)$ and $n|(k-l)$, so
  $\mbox{gcd}(m,n)=1$, and hence $mn|(k-l)$, therefore $k\equiv l\ (\mbox{mod}\
  mn)$. So we have preserved the algebraic structure, and $\phi$ is injective.
  Further, $|\mathbb{Z}_m\times\mathbb{Z}_n|=|\mathbb{Z}_mn|$, so we have
  surjectivity.
  
  Trivially, $\phi(\overline{k}+\overline{l})=
  \phi(\overline{k})+\phi(\overline{l})$ and
  $\phi(\overline{k}\overline{l})=\phi(\overline{k})\phi(\overline{l})$, so we
  have a homomorphism, and so $\mathbb{Z}_m\times\mathbb{Z}_n
  \cong\mathbb{Z}_mn$ when $m$ and $n$ are co-prime. \qed
\end{proof}

If $G$ is a finite group, and for all $g_1, g_2\in G$, $g_1 \circ g_2=g_2\circ
g_1$, $G$ is called \Def{abelian}, and is isomorphic to groups with form
$\mathbb{Z}_n$.
\begin{table}[h]
  \begin{center}
  \begin{tabular}{c|c}
		Number of elements in group & Type\\
		\hline
		$p$ prime & $\mathbb{Z}_p$\\
		$4$ & $\mathbb{Z}_4$, $\mathbb{Z}_2\times\mathbb{Z}_2$\\
		$6$ & $\mathbb{Z}_6\cong\mathbb{Z}_2\times\mathbb{Z}_3$\\
		$8$ & $\mathbb{Z}_8$, $\mathbb{Z}_2\times\mathbb{Z}_2\times\mathbb{Z}_2$,
		$\mathbb{Z}_2\times\mathbb{Z}_4$\\
		$9$ & $\mathbb{Z}_9$, $\mathbb{Z}_3\times\mathbb{Z}_3$
	\end{tabular}\end{center}
\end{table}

Let $G$ be a cyclic group, $g\in G$. The \Def{order} of $g$ is the least
positive integer $r$ such that $g^r=e$. If corresponding elements do not have
the same order, then we do not have an isomorphism; the converse however is not
true.
\begin{example}
  Consider the following examples:
  \begin{enumerate}
    \item $\mathbb{Z}_8^*=\{\overline{1}, \overline{3}, \overline{5},
    \overline{7}\}$, we observe that
    $\overline{3}^2= \overline{5}^2= \overline{7}^2= \overline{1}$, so
    $\mathbb{Z}_8^* \cong\mathbb{Z}_2\times\mathbb{Z}_2$.
    \item $\mathbb{Z}_9^*=\{\overline{1}, \overline{2}, \overline{4},
    \overline{5}, \overline{7}, \overline{8}$, and
    $\mathbb{Z}_9^*\cong\mathbb{Z}_6$ because it is the group with six elements.
    \item $\mathbb{Z}_15^*$ has eight elements, and the order 2 elements are
    $\overline{4}, \overline{11}, \overline{14}$, whilst the order 4 elements
    are $\overline{2}, \overline{7}, \overline{8}, \overline{13}$, and it may be
    seen that $\mathbb{Z}_15^*\cong\mathbb{Z}_2\times\mathbb{Z}_4$.
  \end{enumerate}
\end{example}

%===============================================================================

\section{Permutations}

A \Def{permutation} is a re-arrangement of an order collection of objects.
Consider the set $C_n=\{1,2,\cdots n\}$. A permutation $\sigma$ may be viewed as
a bijective function $\sigma$ from $C_n$ to itself.

\begin{proposition}
  There are $n!$ distinct permutations of $C_n$. \qedwhite
\end{proposition}
In terms of notation, we write
\begin{equation*}
  \sigma=\left\{\begin{matrix}1 & 2 & 3 & 4 & 5\\
  5 & 4 & 3 & 2 & 1\end{matrix}\right\}
\end{equation*}
for $1\mapsto5$ etc. Things in he top row are mapped to the bottom row.

Let $S_n$ be the set of permutations of $C_n$. We want $S_n$ to be a group
under composition of functions. Let $\sigma,\tau:C_n\rightarrow C_n$, be
two permutations, then $\sigma\tau$ or $\tau\sigma$ is also a permutation.
\begin{example}
  For
  \begin{equation*}
    \sigma=\left\{\begin{matrix}1 & 2 & 3 & 4 & 5\\
    5 & 4 & 3 & 2 & 1\end{matrix}\right\},\qquad
    \tau=\left\{\begin{matrix}1 & 2 & 3 & 4 & 5\\
    2 & 3 & 4 & 5 & 1\end{matrix}\right\},
  \end{equation*}
  we have
  \begin{align*}
    \sigma\tau &=\left\{\begin{matrix}1 & 2 & 3 & 4 & 5\\
    5 & 4 & 3 & 2 & 1\end{matrix}\right\}
    \left\{\begin{matrix}1 & 2 & 3 & 4 & 5\\
    2 & 3 & 4 & 5 & 1\end{matrix}\right\}\\
    &=\left\{\begin{matrix}2 & 3 & 4 & 5 & 1\\
    4 & 3 & 2 & 1 & 5\end{matrix}\right\}
    \left\{\begin{matrix}1 & 2 & 3 & 4 & 5\\
    5 & 4 & 3 & 2 & 1\end{matrix}\right\}\\
    &=\left\{\begin{matrix}1 & 2 & 3 & 4 & 5\\
    4 & 3 & 2 & 1 & 5\end{matrix}\right\}.
  \end{align*}
\end{example}
This is \Def{permutation multiplication}, by arranging top/bottom
line accordingly.
\begin{proposition}
  $S_n$ is a group under multiplication of permutations.
\end{proposition}
\begin{proof}
  $S_n$ is closed, composition of functions is associative, and the identity is
  the obvious one. We obtain the inverse permutation $\sigma^{-1}$ by swapping
  the two rows of $\sigma$. \qed
\end{proof}

Consider $S_3$. $|S_3|=3!=6$, so it may be isomorphic to
$\mathbb{Z}_6$. However, we notice that $S_3$ is non-abelian, so it is a
distinct group class. In general, for $n\geq3$, $S_n$ is non-abelian.

%-------------------------------------------------------------------------------

\subsection{Cycles}

A \Def{cycle} on a subset of $C_n$ is a sequence $(a_1,a_2,\cdots a_k)$
of distinct elements of $C_n$, with $k\leq n$. This is a permutation where
\begin{equation*}
  a_1\mapsto a_2\mapsto\cdots\mapsto a_k-1\mapsto a_k\mapsto a_1,\qquad
  r\mapsto r
\end{equation*}
for other values now in the cycle. This is a \Def{$k$-cycle}, denoted
$(a_1 a_2 \cdots a_k)$, as it is made of $k$ elements. Cycles can be written
in several ways:
\begin{equation*}
  (a_1 a_2 \cdots a_k)=\cdots=(a_i a_{i+1} \cdots a_k a_1 \cdots a_{i-1}).
\end{equation*}
Two cycles are \Def{disjoint} if they have no moving elements in common;
for example, $(2517)$ and $(634)$ are disjoint, but $(2517)$ and $(654)$
are not.
\begin{lemma}
  If $\sigma$ and $\tau$ are disjoint cycles, then $\sigma\tau=\tau\sigma$.
\end{lemma}
\begin{proof}
  Moving distinct elements means order of permutation does not matter. \qed
\end{proof}
\begin{theorem}
  Every permutation is an unique produce of disjoint cycles.
\end{theorem}
\begin{proof}
  Let $\sigma:C_n\rightarrow C_n$ be a permutation. Choose $a\in\mathbb{Z}$,
  $1\leq a\leq n$, and let $\sigma^i(a)$ be $\sigma$ applied to $a$ $i$ times
  (so $\sigma^0(a)=a)$. Consider the sequence
  \begin{equation*}
    a,\ \sigma(a),\ \sigma^2(a),\ \cdots\ \sigma^i(a),\ \cdots.
  \end{equation*}
  $C_n$ is finite, so sequence will eventually repeat itself, and there is a
  first time where $\sigma^r(a)=\sigma^s(a)$, with $r<s$. Suppose $r>0$, then
  $\sigma(\sigma^{r-1}(a))=\sigma(\sigma^{s-1}(a))$, but $\sigma$ is bijective,
  which implies $\sigma^{r-1}(a)=\sigma^{s-1}(a)$; thus we have a contradiction,
  and $r=0$. 
  
  Now, let
  \begin{equation*}
    \gamma(a)=\left(a\ \sigma(a)\ \sigma^2(a)\ \cdots\ \sigma^{s-1}(a)\right)
  \end{equation*}
  be a cycle. We construct $\gamma_1=\gamma(a_1)$, a cycle that starts with
  $a_1=1$. If $\gamma_1=\sigma$, we have what we want, otherwise, there is
  a least number $a_2\in\gamma_1$, and we construct $\gamma_2=\gamma(a_2)$,
  a cycle starting with $a_2$. Now, $\gamma_1$ and $\gamma_2$ are disjoint
  by assumption; if $\gamma_1\gamma_2=\sigma$ then we are done. Otherwise we
  repeat the process, and since $C_n$ is fnite, there is a finite collection
  of $k$ where $\gamma_1\gamma_2\cdots\gamma_k=\sigma$. This is essentially
  unique because whenever we have a number $a$, it is automatically in a cycle
  of its own. \qed
\end{proof}
\begin{example}
  \begin{align*}
    \sigma&=\left\{\begin{matrix}
    1 & 2 & 3 & 4 & 5 & 6 & 7 & 8\\
    5 & 7 & 3 & 8 & 2 & 4 & 1 & 6\end{matrix}\right\}=(1527)(3)(486),\\
    \tau&=\left\{\begin{matrix}
    1 & 2 & 3 & 4 & 5 & 6 & 7 & 8 & 9\\
    6 & 3 & 5 & 9 & 1 & 2 & 8 & 7 & 4\end{matrix}\right\}=(16235)(49)(78).
  \end{align*}
  Usually trivial cycles are omitted, so $\sigma=(1527)(486)$.
\end{example}
\begin{example}
  To multiply cycles, consider $\sigma=(135)(48)$ and $\tau=(3218)(46)(57)$, 
  then
  \begin{equation*}
    \sigma\tau=(135)(48)(3218)(46)(57),
  \end{equation*}
  and sending $1$ through from the right, we see that $1\rightarrow8\rightarrow4
  \rightarrow4$, and $4\rightarrow6\rightarrow6$, etc. Doing this for all 
  numbers, we see that $\sigma\tau=(146857)(23)$.
\end{example}
\begin{lemma}
  Let $(a_1\cdots a_k)$ be a $k$-cycle, then
  \begin{equation*}
    (a_1\cdots a_k) = (a_1 a_k)(a_1 a_{k-1})\cdots(a_1 a_2),
  \end{equation*}
  and it is trivial to check this. \qedwhite
\end{lemma}
A $2$-cycle is called a \Def{transposition}. From this, we can deduce
the following:
\begin{theorem}
  Every permutation is a product of transpositions, which follows from the fact 
  that each permutation is a product of disjoint cycles, and every cycle is a 
  produce of transpositions.
\end{theorem}

%-------------------------------------------------------------------------------

\subsection{Cycle types}

Every permutation is a product of disjoint cycles,
$\sigma=\gamma_1\cdots\gamma_r$, say. Suppose teh cycle $\gamma_i$ has
length $k_i$. The unordered sequence of numbers $k_1,k_2\cdots k_r$ is the
\Def{cycle type} of $\sigma$. For example, $(123)(45)$ has type $(3,2)$,
and $(12)(34)(567)$ has type $3,2,3$.
\begin{proposition}
  A permutation of cycle type $k_1,\cdots k_r$ may be expressed as a product of 
  $(k_1+\cdots+k_r)-r$ transpositions. \qedwhite
\end{proposition}
The \Def{parity} of this number $(k_1+\cdots+k_r)-r$ is a property of the 
permutation.
\begin{theorem}[Matrix determinants]
  For a $n\times n$ matrix $\mathsf{A}$,
  \begin{equation*}
    |\mathsf{A}|=\sum_{\sigma\in S_{n}}\epsilon(\sigma)a_{1,\sigma(1)}\cdots
    a_{n,\sigma(n)},
  \end{equation*}
  where $\epsilon(\sigma)$ depends on the parity of the permutation $\sigma$. 
  This is the real definition of the determinant of a matrix. \qedwhite
\end{theorem}

\begin{theorem}
  Given a permutation $\sigma$ written in two ways, one as a product of $r$
  transpositions, the other as a product of $s$ transpositions, $r$ and $s$
  will have the same parity. \qedwhite
\end{theorem}
The \Def{order} of a permutation is the least amount of times the
permutation is composed with itself to get back to the identity. A $k$-cycle has 
order $k$.
\begin{theorem}
  Let $\sigma=\gamma_1\cdots\gamma_k$, $\gamma_i$ disjoint from $\gamma_j$
  for $i\neq j$, and for each $i$, $\gamma_i$ has length $r_i$. Then the order
  of $\sigma$ is $\mbox{lcm}\{r_1,\ r_2, \cdots,\ r_k\}$.
\end{theorem}
\begin{proof}
  Let $\sigma^t=\gamma_1^t \gamma_2^t \cdots \gamma_k^t$. For $\sigma^t=e$,
  $r_i |t$ for all $i$, and the lowest such $t$ is the lowest common multiple of 
  the un-ordered set $\{r_1,\ r_2,\ \cdots,\ r_k\}$. \qed
\end{proof}

%===============================================================================

\section{More on groups}

Here, we write group binary operation as $g\circ h=gh$.

%-------------------------------------------------------------------------------

\subsection{Subgroups}

A \Def{subgroup} of a group $G$ is a subset $H\subseteq G$ such that $H$ 
is also a group under the same group operation as $G$; we write $H\leq G$.
\begin{lemma}
  If $H\leq G$, $e_H=e_G$ and $h^{-1}_H=h^{-1}_G$. Also, for $H\neq\emptyset$ 
  and $H\subseteq G$, $H\leq G$ iff for all $h_1,h_2 \in H$, $h_1 h_2^{-1}
  \in H$.
\end{lemma}
\begin{proof}
  Let $e_h$ be the identity in $H$, and note that $e_H = e_H e_H$. Now, $e_H$ 
  will have inverse $e_H^{-1}$ in $G$, so
  \begin{equation*}
    e_G = e_H^{-1} e_H = e_H^{-1} e_H e_H = e_G e_H = e_H
  \end{equation*}
  as required. Similar,y suppose $h$ has inverse $h^{-1}_G$ in $G$ and 
  $h^{-1}_H$ in $H$, then
  \begin{equation*}
    h^{-1}_G=h^{-1}_G e=h^{-1}_G(h h^{-1}_H) = (h^{-1}_G h) h^{-1}_H =
    e h^{-1}_H = h^{-1}_H.
  \end{equation*}
  
  Since $G$ is associative by assumption, and $H\subseteq G$, $H$ inherits 
  associativity. Since $H\neq\emptyset$, there exists $h\in H$. By previous 
  part, $h h^{-1}=e\in H$, so the identity exists in $H$, and thus the inverse 
  exists also in $H$. For $h,g\in H$, $g^{-1}\in H$, then $h(g^{-1})^{-1}=
  hg\in H$, so we have closure, and thus $H$ is a group. \qed
\end{proof}

\begin{example}
  \begin{enumerate}
    \item Let $\mathbb{C}^*$ be the group of non-zero complex numbers under 
    multiplication, and let
    \begin{equation*}
      H=\{\ex^{2\pi\zi k/n}\ |\ 0\leq k<n,\ n\geq 2\}.
    \end{equation*}
    Since $\ex^{2\pi\zi k_1/n}(\ex^{2\pi\zi k_2/n})^{-1}=
    \ex^{2\pi\zi(k_1 - k_2)/n}\in H$ taking $k_1 - k_2$ in mod $n$, 
    $H\leq\mathbb{C}^*$ by previous lemma. (In fact $H\cong\mathbb{Z}_n$).
    
    \item For $S_n$ the group of permutations, let $A_n$ be the subset of all 
    even permutations in $S_n$. ($A_n$ is known as the \Def{alternating 
    group}.) To show $A_n\leq S_n$, we have that
    \begin{equation*}
      \sigma=(a_1 b_1)\cdots(a_k b_k),\qquad\Rightarrow\qquad
      \sigma^{-1}=(a_k b_k)\cdots(a_1 b_1).
    \end{equation*}
    We see the parity of $\sigma$ and $\sigma^{-1}$ are equal, so for any two 
    permutations $\sigma,\tau\in A_n$, $\sigma\tau^{-1}$ is an even permutation, 
    and thus $A_n\leq S_n$. (Note also that $|A_n|=n!/2$.)
  \end{enumerate}
\end{example}

For all $G$, $\{e\}$ and $G$ are also subgroups of $G$, known as the
\Def{improper subgroups} of $G$.

%-------------------------------------------------------------------------------

\subsection{Order and cosets}

The \Def{order} of an element $g\in G$, denoted $|g|$, is the least
positive integer $n$ such that $g^n=e$ if $n<\infty$, otherwise they are of
infinite order.
\begin{proposition}
  Let $g\in G$, with $|g|=n<\infty$. Then the set $\langle g\rangle]\{g^k\ |\ 0
  \leq k<n\}$ is a subgroup of $G$, known as the \Def{cyclic group} 
  generated by $g$.
\end{proposition}
\begin{proof}
  Let $t\in\mathbb{Z}^+$, then $t=qn+r$, $0\leq r<n$. So $g^t = g^{qn+r} = 
  (g^n)^q g^r = e^q g^r = g^r$, so we have closure. Associativity follows since 
  $\langle g\rangle\subseteq G$. Identity exists by definition, and $(g^k)^{-1}
  =g^{n-k}$ is the inverse. \qed
\end{proof}

The order of a group is the number of elements of $G$, denoted $|G|$.
\begin{theorem}
  Any group of prime order is cyclic. Any non-identity element can be the 
  generator of the group.
\end{theorem}
To proof this, we make use of the following theorem:
\begin{theorem}[Lagrange]
  If $H\leq G$, then $|H|$ divides $|G|$.
\end{theorem}
\begin{proof}[Proof of theorem above]
  Let $g\in G$, and $g\neq e$. By Lagrange's theorem, $|\langle g\rangle|$ 
  divides $|G|=p$, and since $p$ is prime and $g\neq e$, $|\langle g\rangle|=p$, 
  and $\langle g\rangle=G$. \qed
\end{proof}

To proof Lagrange's theorem, we make use of the idea of \Def{cosets}. For
$H\leq G$ and $g\in G$, the \Def{right coset of $H$ in G} is the
set $gH=\{gh\ |\ h\in H\}$, whilst the \Def{left coset of $H$ in G}
is the set $Hg=\{hg\ |\ h\in H\}$.
\begin{lemma}
  We have the following:
  \begin{enumerate}
    \item Let $X$ be a finite subset of a group $G$, and $g\in G$. Define $gX$ 
    and $Xg$ like cosets, then $|gX|=|Xg|=|X|$.
    \item If $gH\cap g'H\neq\emptyset$, then $gH=g'H$, and similarly for right 
    cosets.
    \item The union of all left cosets of $G$ in $G$ is the whole of $G$, and 
    similarly for right cosets.
  \end{enumerate}
\end{lemma}
\begin{proof}
  In order:
  \begin{enumerate}
    \item Let $x\neq x'$, $x,x'\in X$. If $gx=gx'$, then $g^{-1}gx=g^{-1}gx'$ 
    which implies $x=x'$, and we have a contradiction, thus $x=x'$. the list is 
    still unchanged in terms of size, so $|gX|=|X|$ and similarly for $|Xg|$.
    \item Assuming $gH\cap g'H\neq\emptyset$. Let $x\in gH\cap g'H$, then there 
    exists $h,h'\in H$ such that $x=gh=g'h'$, so
    \begin{equation*}
      g=ge=(gh)h^{-1}=(g'h')h^{-1}.
    \end{equation*}
    Let $y\in gH$, $y=gh''$, then $(g'h'h^{-1})h''=g'(h'h^{-1}h'')$, and since 
    $H$ is a group and is closed, $h'h^{-1}h''\in H$, so $y\in g'H$, therefore 
    $gH\subseteq g'H$. Similar arguments give $g'H\subseteq gH$, so $gH=g'H$.
    \item Let $g\in G$, then $g=ge=eg$, and since $e\in H$, $g\in gH$ and 
    $g\in Hg$ for all $g\in G$, so the union of all cosets covers all of $G$. \qed
  \end{enumerate}
\end{proof}

In summary:
\begin{itemize}
  \item the size of a coset is the same as the set it is being acted on;
  \item all left cosets are either equal or disjoint, and similarly with right 
  cosets;
  \item the union of all cosets is the group;
  \item left coset is equal to right coset if the group being acted on is 
  abelian.
\end{itemize}

\begin{proof}[Proof of Lagrange's theorem]
  $|G|$ is equal to the number of cosets that are distinct, multiplied by the 
  size of the cosets (which is common to all cosets). Now, $H=eH$, so the common 
  coset size is $|H|$, and $|H|$ divides $|G|$ as required. \qed
\end{proof}
Note that it didn't matter whether we used right or left cosets, so the number 
of right cosets is equal to the number of left cosets.

\begin{corollary}
  If $g\in G$, then $|g|$ divides $|G|$.
\end{corollary}
\begin{proof}
  Let $H=\langle g\rangle$, then $|g|=|\langle g\rangle|$. Since $|H|$ divides 
  $|G|$, $|g|$ divides $|G|$. \qed
\end{proof}

The \Def{index} $|G:H|$ is the number of left (right) cosets of $H$ in $G$ 
that are distinct. So Lagrange's theorem may be restated as
\begin{equation*}
  |G|=|G:H|\cdot|H|.
\end{equation*}
\begin{example}
  We note that $A_n\leq S_n$. Consider the transposition $(12)\not\in A_n$. Let 
  $\sigma$ be an odd permutation, so that $(12)\sigma\in A_n$. Then observe that 
  $(12)(12)\sigma=e\sigma=\sigma\in(12)A_n$, so all odd permutations are in the 
  coset $(12)A_n$.
  
  A permutation is either even or odd, hence
  \begin{equation*}
    S_n=A_n\cup(12)A_n,\qquad |S_n:A_n|=2\qquad\Rightarrow\qquad |A_n|=n!/2
  \end{equation*}
  because $|S_n|=n!$. This also shows that there are as many even permutations 
  in $S_n$ as odd permutations.
\end{example}

%-------------------------------------------------------------------------------

\subsection{Isomorphisms}

A group $G$ is isomorphic to $H$ if there exists $\phi: G\rightarrow H$ where 
$\phi$ is a:
\begin{enumerate}
  \item \Def{homomorphism} -- For all $g_1,g_2\in G$, $\phi(g_1 g_2) 
  = \phi(g_1)\phi(g_2)$;
  \item \Def{epimorphism} (surjectivity) -- For all $h\in H$, there exists 
  $g\in G$ such that $\phi(g)=h$;
  \item \Def{monomophism} (injectivity) -- For all $g_1,g_2\in G$, $\phi
  (g_1)=\phi(g_2)$ implies that $g_1=g_2$.
\end{enumerate}
The first property says that the group structure is perserved, and the other two says that $\phi$ is a bijection.

\begin{lemma}
  $\phi:G\rightarrow H$ is a homomorphism iff:
  \begin{enumerate}
    \item For all $\phi(g)=e_H$, $g=e_G$;
    \item for all $g\in G$, $\phi(g^{-1})=(\phi(g))^{-1}$.
  \end{enumerate}
\end{lemma}
\begin{proof}
  Let $h=\phi(e_G)$, then
  \begin{enumerate}
    \item $hh=\phi(e_G)\phi(e_G)=\phi(e_G e_G)=\phi(e_G)=h$, so $h=e_H$.
    \item $e_h=\phi(e_G)=\phi(gg^{-1})=\phi(g)\phi(g^{-1})$, so $\phi(g^{-1})=
    (\phi(g))^{-1}$.
  \end{enumerate}
  \qed
\end{proof}

\begin{example}
  Examples of homomorphisms include
  \begin{equation*}
    \phi: S_3\rightarrow\{\pm1\},\qquad \phi(\sigma)=\begin{cases}
    +1, &\textnormal{if $\sigma$ even},\\
    -1, &\textnormal{if $\sigma$ odd},\end{cases}
  \end{equation*}
  \begin{equation*}
    \phi:\mathbb{Z}\rightarrow\mathbb{Z},\qquad \phi(n)=kn,
  \end{equation*}
  \begin{equation*}
    \phi:\mathbb{Z}_3\rightarrow A_3,\qquad \phi(\overline{0})=e,\qquad
    \phi(\overline{1})=(123),\qquad \phi(\overline{2})=(132).
  \end{equation*}
\end{example}
%===============================================================================

\section{Symmetry}

A \Def{symmetry} on an object is a function that sends the object to itself and
preserves the basic structure of the object. For example, for an equilateral
triangle with vertices labelled $1,2,3$, we have
\begin{table}[h]
  \begin{center}
  \begin{tabular}{c|c}
		Symmetry & permutation\\
		\hline
		Reflection, $1,2,3$ invariant & $(23),(13),(12)$\\
		Rotation, $(2\pi/3)^n$ anti-clockwise, $n=0,1,2$ & $e,(132),(123)$
	\end{tabular}\end{center}
\end{table}
In fact, this group it complete as there are no more ways to permute the 
numbers. This forms the \Def{dihedral group} $D_3$.

Now consider the square, and we have symmetries
\begin{table}[h]
  \begin{center}
  \begin{tabular}{c|c}
		Symmetry & symbol\\
		\hline
		Rotation, $(\pi/2)^n$ anti-clockwise, $n=0,1,2,3$ & $e,r,r^2,r^3$\\
		Vertical reflection & $v$\\
		Horizontal reflection & $h$\\
		Leading diagonal reflection & $d_1$\\
		Off-diagonal reflection & $d_2$
	\end{tabular}\end{center}
\end{table}
We see that $\{e,r,r^2,r^3\}$ form a cyclic group with $r$ as the generator,
which appears to be a subgroup of order four in $D_4$. Another thing to
notice is that all reflections are inverses of themselves, so that $\{e,v\}$,
$\{e,h\}$, $\{e,d_1\}$, $\{e,d_2\}$ are also subgroups of $D_4$. Further, it may 
be shown that
\begin{equation*}
  rh=d_1,\qquad r^2 h = v,\qquad r^3 h = d_2,
\end{equation*}
so it seems that we can generate $D_4$ using $r$ and $h$ (or indeed any of the 
reflections together with a rotation).

For a regular $n$-gon, we let $r$ be the rotation by $2\pi/n$, and $h$ to be any 
reflection. These then have the relations
\begin{equation*}
  r^n=e,\qquad h^2=e,\qquad (rh)^2=e,\qquad rh=hr^{-1},
\end{equation*}
and $D_n=\{e,r,\cdots r^{n-1},h,rh\cdots r^{n-1}h\}$ forms a group of order
$2n$. (Note that for a regular $n$-gon, there are $2n$ lines of reflection
although only $n$ of them are distinct.) Further, rotational symmetries form
a cyclic group of order $n$, generated by $r$, which is a subgroup of $D_n$
with index two, whilst reflectional symmetries form a subgroup of order two,
generated by each individual reflection, of index $n$.

\begin{example}
  Find all the subgroups of order four in $D_8$.
  
  The subgroup either has an element of order four, or has identity and three 
  order two elements.
  \begin{enumerate}
    \item Since $|r^2|=4$, $\{e,r^2,r^4,r^6\}\leq D_8$.
    \item All reflections and $r^4$ have order four. A subgroup of this type 
    must contain at least two reflections, $r^i h$ and $r^j h$ say, with 
    ($i>j$). Now,
    \begin{equation*}
      r^i h r^j h=r^i r^{-j} hh=r^{i-j}\neq e,
    \end{equation*}
    so it is a rotation thus $r^4$, which implies that $i=4+j$. Hence the 
    subgroups of this type are
    \begin{equation*}
      \{e,r^4,h,r^4 h\},\qquad \{e,r^4,rh,r^5 h\},\qquad
      \{e,r^4,r^2 h,r^6 h\},\qquad \{e,r^4,r^3 h,r^7 h\}.
    \end{equation*}
  \end{enumerate}
\end{example}

%===============================================================================

%%%%%%%%%%%%%%%%%%%%%%%%%%%%%%%%%%%%%%%%%

% r.5 contents
%\tableofcontents

%\listoffigures

%\listoftables

% r.7 dedication
%\cleardoublepage
%~\vfill
%\begin{doublespace}
%\noindent\fontsize{18}{22}\selectfont\itshape
%\nohyphenation
%Dedicated to those who appreciate \LaTeX{} 
%and the work of \mbox{Edward R.~Tufte} 
%and \mbox{Donald E.~Knuth}.
%\end{doublespace}
%\vfill

% r.9 introduction
% \cleardoublepage

%%%%%%%%%%%%%%%%%%%%%%%%%%%%%%%%%%%%%%%%%
% actual useful crap (normal chapters)
\mainmatter

%\part{Basics (?)}


%\backmatter

%\bibliography{refs}
\bibliographystyle{plainnat}

%\printindex

\end{document}

