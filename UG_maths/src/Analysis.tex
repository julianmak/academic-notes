\documentclass[letter-paper]{tufte-book}

%%
% Book metadata
\title{Analysis 3H}
\author[]{B. S. H. Mithrandir}
%\publisher{Research Institute of Valinor}

%%
% If they're installed, use Bergamo and Chantilly from www.fontsite.com.
% They're clones of Bembo and Gill Sans, respectively.
\IfFileExists{bergamo.sty}{\usepackage[osf]{bergamo}}{}% Bembo
\IfFileExists{chantill.sty}{\usepackage{chantill}}{}% Gill Sans

%\usepackage{microtype}
\usepackage{amssymb}
\usepackage{amsmath}
%%
% For nicely typeset tabular material
\usepackage{booktabs}

%% overunder braces
\usepackage{oubraces}

%% 
\usepackage{xcolor}
\usepackage{tcolorbox}

\newtcolorbox[auto counter,number within=section]{derivbox}[2][]{colback=TealBlue!5!white,colframe=TealBlue,title=Box \thetcbcounter:\ #2,#1}                                                          

\makeatletter
\@openrightfalse
\makeatother

%%
% For graphics / images
\usepackage{graphicx}
\setkeys{Gin}{width=\linewidth,totalheight=\textheight,keepaspectratio}
\graphicspath{{figs/}}

% The fancyvrb package lets us customize the formatting of verbatim
% environments.  We use a slightly smaller font.
\usepackage{fancyvrb}
\fvset{fontsize=\normalsize}

\usepackage[plain]{fancyref}
\newcommand*{\fancyrefboxlabelprefix}{box}
\fancyrefaddcaptions{english}{%
  \providecommand*{\frefboxname}{Box}%
  \providecommand*{\Frefboxname}{Box}%
}
\frefformat{plain}{\fancyrefboxlabelprefix}{\frefboxname\fancyrefdefaultspacing#1}
\Frefformat{plain}{\fancyrefboxlabelprefix}{\Frefboxname\fancyrefdefaultspacing#1}

%%
% Prints argument within hanging parentheses (i.e., parentheses that take
% up no horizontal space).  Useful in tabular environments.
\newcommand{\hangp}[1]{\makebox[0pt][r]{(}#1\makebox[0pt][l]{)}}

%% 
% Prints an asterisk that takes up no horizontal space.
% Useful in tabular environments.
\newcommand{\hangstar}{\makebox[0pt][l]{*}}

%%
% Prints a trailing space in a smart way.
\usepackage{xspace}
\usepackage{xstring}

%%
% Some shortcuts for Tufte's book titles.  The lowercase commands will
% produce the initials of the book title in italics.  The all-caps commands
% will print out the full title of the book in italics.
\newcommand{\vdqi}{\textit{VDQI}\xspace}
\newcommand{\ei}{\textit{EI}\xspace}
\newcommand{\ve}{\textit{VE}\xspace}
\newcommand{\be}{\textit{BE}\xspace}
\newcommand{\VDQI}{\textit{The Visual Display of Quantitative Information}\xspace}
\newcommand{\EI}{\textit{Envisioning Information}\xspace}
\newcommand{\VE}{\textit{Visual Explanations}\xspace}
\newcommand{\BE}{\textit{Beautiful Evidence}\xspace}

\newcommand{\TL}{Tufte-\LaTeX\xspace}

% Prints the month name (e.g., January) and the year (e.g., 2008)
\newcommand{\monthyear}{%
  \ifcase\month\or January\or February\or March\or April\or May\or June\or
  July\or August\or September\or October\or November\or
  December\fi\space\number\year
}


\newcommand{\urlwhitespacereplace}[1]{\StrSubstitute{#1}{ }{_}[\wpLink]}

\newcommand{\wikipedialink}[1]{http://en.wikipedia.org/wiki/#1}% needs \wpLink now

\newcommand{\anonymouswikipedialink}[1]{\urlwhitespacereplace{#1}\href{\wikipedialink{\wpLink}}{Wikipedia}}

\newcommand{\Wikiref}[1]{\urlwhitespacereplace{#1}\href{\wikipedialink{\wpLink}}{#1}}

% Prints an epigraph and speaker in sans serif, all-caps type.
\newcommand{\openepigraph}[2]{%
  %\sffamily\fontsize{14}{16}\selectfont
  \begin{fullwidth}
  \sffamily\large
  \begin{doublespace}
  \noindent\allcaps{#1}\\% epigraph
  \noindent\allcaps{#2}% author
  \end{doublespace}
  \end{fullwidth}
}

% Inserts a blank page
\newcommand{\blankpage}{\newpage\hbox{}\thispagestyle{empty}\newpage}

\usepackage{units}

% Typesets the font size, leading, and measure in the form of 10/12x26 pc.
\newcommand{\measure}[3]{#1/#2$\times$\unit[#3]{pc}}

% Macros for typesetting the documentation
\newcommand{\hlred}[1]{\textcolor{Maroon}{#1}}% prints in red
\newcommand{\hangleft}[1]{\makebox[0pt][r]{#1}}
\newcommand{\hairsp}{\hspace{1pt}}% hair space
\newcommand{\hquad}{\hskip0.5em\relax}% half quad space
\newcommand{\TODO}{\textcolor{red}{\bf TODO!}\xspace}
\newcommand{\na}{\quad--}% used in tables for N/A cells
\providecommand{\XeLaTeX}{X\lower.5ex\hbox{\kern-0.15em\reflectbox{E}}\kern-0.1em\LaTeX}
\newcommand{\tXeLaTeX}{\XeLaTeX\index{XeLaTeX@\protect\XeLaTeX}}
% \index{\texttt{\textbackslash xyz}@\hangleft{\texttt{\textbackslash}}\texttt{xyz}}
\newcommand{\tuftebs}{\symbol{'134}}% a backslash in tt type in OT1/T1
\newcommand{\doccmdnoindex}[2][]{\texttt{\tuftebs#2}}% command name -- adds backslash automatically (and doesn't add cmd to the index)
\newcommand{\doccmddef}[2][]{%
  \hlred{\texttt{\tuftebs#2}}\label{cmd:#2}%
  \ifthenelse{\isempty{#1}}%
    {% add the command to the index
      \index{#2 command@\protect\hangleft{\texttt{\tuftebs}}\texttt{#2}}% command name
    }%
    {% add the command and package to the index
      \index{#2 command@\protect\hangleft{\texttt{\tuftebs}}\texttt{#2} (\texttt{#1} package)}% command name
      \index{#1 package@\texttt{#1} package}\index{packages!#1@\texttt{#1}}% package name
    }%
}% command name -- adds backslash automatically
\newcommand{\doccmd}[2][]{%
  \texttt{\tuftebs#2}%
  \ifthenelse{\isempty{#1}}%
    {% add the command to the index
      \index{#2 command@\protect\hangleft{\texttt{\tuftebs}}\texttt{#2}}% command name
    }%
    {% add the command and package to the index
      \index{#2 command@\protect\hangleft{\texttt{\tuftebs}}\texttt{#2} (\texttt{#1} package)}% command name
      \index{#1 package@\texttt{#1} package}\index{packages!#1@\texttt{#1}}% package name
    }%
}% command name -- adds backslash automatically
\newcommand{\docopt}[1]{\ensuremath{\langle}\textrm{\textit{#1}}\ensuremath{\rangle}}% optional command argument
\newcommand{\docarg}[1]{\textrm{\textit{#1}}}% (required) command argument
\newenvironment{docspec}{\begin{quotation}\ttfamily\parskip0pt\parindent0pt\ignorespaces}{\end{quotation}}% command specification environment
\newcommand{\docenv}[1]{\texttt{#1}\index{#1 environment@\texttt{#1} environment}\index{environments!#1@\texttt{#1}}}% environment name
\newcommand{\docenvdef}[1]{\hlred{\texttt{#1}}\label{env:#1}\index{#1 environment@\texttt{#1} environment}\index{environments!#1@\texttt{#1}}}% environment name
\newcommand{\docpkg}[1]{\texttt{#1}\index{#1 package@\texttt{#1} package}\index{packages!#1@\texttt{#1}}}% package name
\newcommand{\doccls}[1]{\texttt{#1}}% document class name
\newcommand{\docclsopt}[1]{\texttt{#1}\index{#1 class option@\texttt{#1} class option}\index{class options!#1@\texttt{#1}}}% document class option name
\newcommand{\docclsoptdef}[1]{\hlred{\texttt{#1}}\label{clsopt:#1}\index{#1 class option@\texttt{#1} class option}\index{class options!#1@\texttt{#1}}}% document class option name defined
\newcommand{\docmsg}[2]{\bigskip\begin{fullwidth}\noindent\ttfamily#1\end{fullwidth}\medskip\par\noindent#2}
\newcommand{\docfilehook}[2]{\texttt{#1}\index{file hooks!#2}\index{#1@\texttt{#1}}}
\newcommand{\doccounter}[1]{\texttt{#1}\index{#1 counter@\texttt{#1} counter}}

\newcommand{\studyq}[1]{\marginnote{Q: #1}}

\hypersetup{colorlinks}% uncomment this line if you prefer colored hyperlinks (e.g., for onscreen viewing)

% Generates the index
\usepackage{makeidx}
\makeindex

\setcounter{tocdepth}{3}
\setcounter{secnumdepth}{3}

%%%%%%%%%%%%%%%%%%%%%%%%%%%%%%%%%%%%%%%%%%%%%%%%%%%%%%%%%%%%%%
% custom commands

\newtheorem{theorem}{\color{pastel-blue}Theorem}[section]
\newtheorem{lemma}[theorem]{\color{pastel-blue}Lemma}
\newtheorem{proposition}[theorem]{\color{pastel-blue}Proposition}
\newtheorem{corollary}[theorem]{\color{pastel-blue}Corollary}

\newenvironment{proof}[1][Proof]{\begin{trivlist}
\item[\hskip \labelsep {\bfseries #1}]}{\end{trivlist}}
\newenvironment{definition}[1][Definition]{\begin{trivlist}
\item[\hskip \labelsep {\bfseries #1}]}{\end{trivlist}}
\newenvironment{example}[1][Example]{\begin{trivlist}
\item[\hskip \labelsep {\bfseries #1}]}{\end{trivlist}}
\newenvironment{remark}[1][Remark]{\begin{trivlist}
\item[\hskip \labelsep {\bfseries #1}]}{\end{trivlist}}

\hyphenpenalty=5000

% more pastel ones
\xdefinecolor{pastel-red}{rgb}{0.77,0.31,0.32}
\xdefinecolor{pastel-green}{rgb}{0.33,0.66,0.41}
\definecolor{pastel-blue}{rgb}{0.30,0.45,0.69} % crayola blue
\definecolor{gray}{rgb}{0.2,0.2,0.2} % dark gray

\xdefinecolor{orange}{rgb}{1,0.45,0}
\xdefinecolor{green}{rgb}{0,0.35,0}
\definecolor{blue}{rgb}{0.12,0.46,0.99} % crayola blue
\definecolor{gray}{rgb}{0.2,0.2,0.2} % dark gray

\xdefinecolor{cerulean}{rgb}{0.01,0.48,0.65}
\xdefinecolor{ust-blue}{rgb}{0,0.20,0.47}
\xdefinecolor{ust-mustard}{rgb}{0.67,0.52,0.13}

%\newcommand\comment[1]{{\color{red}#1}}

\newcommand{\dy}{\partial}
\newcommand{\ddy}[2]{\frac{\dy#1}{\dy#2}}

\newcommand{\ab}{\boldsymbol{a}}
\newcommand{\bb}{\boldsymbol{b}}
\newcommand{\cb}{\boldsymbol{c}}
\newcommand{\db}{\boldsymbol{d}}
\newcommand{\eb}{\boldsymbol{e}}
\newcommand{\lb}{\boldsymbol{l}}
\newcommand{\nb}{\boldsymbol{n}}
\newcommand{\tb}{\boldsymbol{t}}
\newcommand{\ub}{\boldsymbol{u}}
\newcommand{\vb}{\boldsymbol{v}}
\newcommand{\xb}{\boldsymbol{x}}
\newcommand{\wb}{\boldsymbol{w}}
\newcommand{\yb}{\boldsymbol{y}}
\newcommand{\zb}{\boldsymbol{z}}

\newcommand{\Xb}{\boldsymbol{X}}

\newcommand{\ex}{\mathrm{e}}
\newcommand{\zi}{{\rm i}}

\newcommand\Real{\mbox{Re}} % cf plain TeX's \Re and Reynolds number
\newcommand\Imag{\mbox{Im}} % cf plain TeX's \Im

\newcommand{\zbar}{{\overline{z}}}

\newcommand\Def[1]{\textbf{#1}}

\newcommand{\qed}{\hfill$\blacksquare$}
\newcommand{\qedwhite}{\hfill \ensuremath{\Box}}

%%%%%%%%%%%%%%%%%%%%%%%%%%%%%%%%%%%%%%%%%%%%%%%%%%%%%%%%%%%%%%
% some extra formatting (hacked from Patrick Farrell's notes)
%  https://courses.maths.ox.ac.uk/node/view_material/4915
%

% chapter format
\titleformat{\chapter}%
  {\huge\rmfamily\itshape\color{pastel-red}}% format applied to label+text
  {\llap{\colorbox{pastel-red}{\parbox{1.5cm}{\hfill\itshape\huge\color{white}\thechapter}}}}% label
  {1em}% horizontal separation between label and title body
  {}% before the title body
  []% after the title body

% section format
\titleformat{\section}%
  {\normalfont\Large\itshape\color{pastel-green}}% format applied to label+text
  {\llap{\colorbox{pastel-green}{\parbox{1.5cm}{\hfill\color{white}\thesection}}}}% label
  {1em}% horizontal separation between label and title body
  {}% before the title body
  []% after the title body

% subsection format
\titleformat{\subsection}%
  {\normalfont\large\itshape\color{pastel-blue}}% format applied to label+text
  {\llap{\colorbox{pastel-blue}{\parbox{1.5cm}{\hfill\color{white}\thesubsection}}}}% label
  {1em}% horizontal separation between label and title body
  {}% before the title body
  []% after the title body

%%%%%%%%%%%%%%%%%%%%%%%%%%%%%%%%%%%%%%%%%%%%%%%%%%%%%%%%%%%%%%%%%%%%%%%%%%%%%%%%

\begin{document}

% Front matter
%\frontmatter

% r.3 full title page
%\maketitle

% v.4 copyright page

\chapter*{}

\begin{fullwidth}

\par \begin{center}{\Huge Analysis 3H}\end{center}

\vspace*{5mm}

\par \begin{center}{\Large typed up by B. S. H. Mithrandir}\end{center}

\vspace*{5mm}

\begin{itemize}
  \item \textit{Last compiled: \monthyear}
  \item Adapted from notes of D. Sch\"utz, Durham
  \item This was part of the Analysis 3H module elective. This is a course on
  real analysis, touching on metric spaces, tangent spaces, vector fields,
  manifolds, and differential forms.
  \item[]
  \item \TODO diagrams
\end{itemize}

\par

\par Licensed under the Apache License, Version 2.0 (the ``License''); you may not
use this file except in compliance with the License. You may obtain a copy
of the License at \url{http://www.apache.org/licenses/LICENSE-2.0}. Unless
required by applicable law or agreed to in writing, software distributed
under the License is distributed on an \smallcaps{``AS IS'' BASIS, WITHOUT
WARRANTIES OR CONDITIONS OF ANY KIND}, either express or implied. See the
License for the specific language governing permissions and limitations
under the License.
\end{fullwidth}


%===============================================================================

\chapter{Metric spaces}

%-------------------------------------------------------------------------------

\section{Basic notions}

The field of real numbers $\mathbb{R}$ is a totally ordered field which also
satisfies the \textbf{completeness} axiom, i.e. a non-empty bounded set $A
\subseteq \mathbb{R}$ has a \textbf{supremum} and/or an \textbf{infimum}. The
supremum of $A\subseteq\mathbb{R}$ is a real number $s$ where $a\leq s$ for all
$a\in A$. If $m$ is also such that $a \leq m$ for $a \in A$, then $s\leq m$,
denoted $\sup A$. The infimum of $A$ is where the inequalities signs are
swapped, denoted $\inf A$.

\begin{lemma}\label{lem:intervals}
  Let $I_n = [a_n, b_n]$ be a sequence of closed intervals such that $a_n \leq
  a_{n+1} < b_{n+1} \leq b_n$ for all $n\geq 1$, then $\cap_{n=1}^\infty I_n$ is
  non-empty. \qedwhite
\end{lemma}

\begin{proof}
  Let $a = \sup\{ a_n \}$. Since $a_n \leq b_1$ for all $n$ exists by
  completeness axiom, $a_n \leq b_k$ for any value of $n$ and $k$, and so $a\leq
  b_k$. Hence $a_k \leq a \leq b_k$ for all $k$, and that $a \in
  \cap_{n=1}^\infty I_n$.
\end{proof}

Let $M$ be a set. A function $d : M\times M \to [0, \infty)$ is called a
\textbf{metric} on $M$ if
\begin{enumerate}
  \item $d(x, y) = 0$ iff $x = y$;
  \item $d(x, y) = d(y, x)$ for all $x,y\in M$;
  \item $d(x, z) \leq d(x, y) + d(y, z)$ for all $x, y, z \in M$.
\end{enumerate}
The pair $(M, d)$ is then called a \textbf{metric space}. It is easy to see
any $N\subseteq M$ is also a metric space using the same $d$.

\begin{example}
  \begin{enumerate}
    \item On $\mathbb{R}$, $d(x,y) = |y - x|$ gives a metric.
    \item On $\mathbb{R}^2$, $d_1 (\xb, yb) = |y_1 - x_1| + |y_2 - x_2|$ is also a
    metric, but notice that, for example, $d_1( (1, 1), (0, 0) ) = 2$ as opposed
    to the expected distance of $\sqrt{2}$.
  \end{enumerate}
\end{example}

The standard (Euclidean) metric in $\mathbb{R}^2$ is given by
\begin{equation*}
  d(\xb, \yb) = \sqrt{|x_1 - y_1|^2 + |x_2 - y_2|^2}.
\end{equation*}

Let $V$ be a real vector space. An \textbf{inner product} on $V$ is a
function $(\cdot, \cdot) : V\times V\to \mathbb{R}$ that, for all $\xb, \yb\in
V$, satisfies the following:
\begin{itemize}
  \item linearity in the first factor;
  \item $(\xb, \yb) = (\yb, \xb)$;
  \item $(\xb, \xb) \geq 0$ and is zero iff $\xb = 0$.
\end{itemize}
\begin{example}
  \begin{enumerate}
    \item For $V = \mathbb{R}^n$, the standard inner product is given by $(\xb,
    \yb) = x_i y_i$ (where Einstein notation is understood). If $\boldsymbol{A}$
    is a symmetric matrix, then $(\xb, \yb) = \xb^T \boldsymbol{A} \yb$ is an
    inner product if all eigenvalues of $\boldsymbol{A}$ are positive.
    
    \item For $V = C[a, b]$, $(f, g) = \int_a^b f(x) g(x)\, \mathrm{d}x$ is an
    inner product since $V$ is a vector space of continuous functions, and the
    only function that is everywhere zero and continuous is $f(x) = 0$ for all
    $x\in[a, b]$.
  \end{enumerate}
\end{example}

\begin{theorem}[Cauchy--Schwartz inequality]
  Let $V$ be a real vector space, and $(\cdot, \cdot)$ an inner product on $V$.
  Then
  \begin{equation*}
    \left|(\xb, \yb)\right| \leq \|\xb\| \cdot \|\yb\|,
  \end{equation*}
  where $\|\cdot\|$ is the standard Euclidean norm of the vector, and there is
  equality iff $\xb = \lambda \yb$ for some $\lambda \in \mathbb{R}$.
\end{theorem}

\begin{proof}
  Note that $(\xb, \boldsymbol{0}) = (\xb, \xb - \xb) = (\xb, \xb) - (\xb, \xb)
  = \boldsymbol{0}$, so we may assume that $\yb \neq \boldsymbol{0}$. Then, with
  $\lambda = -(\xb, \yb) / \|\yb\|^2$,
  \begin{align*}
    0 \leq (\xb + \lambda \yb, \xb + \lambda \yb) 
      & = \| \xb \|^2 + 2\lambda(\xb, \yb) + \lambda^2 \| \yb \|^2\\
      & = \| \xb \|^2 - \frac{(\xb, \yb)^2}{\| \yb \|^2}.
  \end{align*}
  So $(\xb, \yb)^2 \leq \| \xb \|^2 \| \yb \|^2$ and the result follows. \qed
\end{proof}

\begin{lemma}
  Let $V$ be a real vector space with inner product $(\cdot, \cdot)$. Then $d: V
  \times V \to [0, \infty)$ with $d(\xb, \yb) = \| \xb - \yb\|$ gives a metric
  on $V$.
\end{lemma}

\begin{proof}
  Clearly $d(\xb, \xb) = 0$ and is symmetric, so we just need to check the
  triangle inequality. By Cauchy--Schwartz,
  \begin{align*}
    \| \ab + \bb\| &= \sqrt{\|\ab\|^2 + 2(\ab, \bb) + \| \bb\|^2} \\
      &\leq \sqrt{\|\ab\|^2 + 2\|\ab\| \|\bb\| + \| \bb\|^2} \\
      &\leq \|\ab\| + \|\bb\|,
  \end{align*}
  as required. \qed
\end{proof}

Let $f: M \to N$ be a function metric metric spaces $(M, d_M)$ and $(N, d_N)$.
For $a \in M$, $f$ is \textbf{continuous at $a$} if, for all $\epsilon > 0$,
there exists $\delta >0$ such that $d_N(f(a), f(x)) < \epsilon$ for all $x\in M$
when $d_M(a, x) < \delta$.

%-------------------------------------------------------------------------------

\section{Sequences and Cauchy sequences}

Let $M$ be a metric space. A \textbf{sequence} $(a_n)$ in $M$ consists of
elements $a_n \in M$ for all $n\in\mathbb{N}$. Let $a \in M$, and $(a_n)$
\textbf{converges to $a$} if, for all $\epsilon > 0$, $d(a_n, a) < \epsilon$
for some all $n \geq n_0$. We write $\lim_{n\to\infty} a_n = a$. The sequence
$(a_n)$ is called \textbf{convergent} if there exists $a\in M$ where $a_n \to
a$.

\begin{lemma}
  Let $f : M \to N$ be a function between metric spaces and $a \in M$. The
  function $f$ is continuous at $a \in M$ iff $f(a_n) \to f(a)$ for $(a_n) \in
  M$ with $a_n \to a$. (Note that $f(a_n)$ is a sequence in $N$.)
\end{lemma}

\begin{proof}
  Assume that $f$ is continuous at $a \in M$, and let $(a_n)$ be a sequence with
  $a_n \to a$. By continuity, for any $\epsilon > 0$, there exists $\delta > 0$
  such that, for $d(a, y) < \delta$, $d(f(a), f(y)) <\epsilon$ for arbitrary $y
  \in M$. Choose $n_0 \geq 0$ such that $d(a_n, a) < \delta$ for all $n\geq
  n_0$, then this implies $d(f(a_n), f(a)) < \epsilon$, and thus $f(a_n) \to
  f(a)$ as required.
  
  On the other hand, assume $f(a_n) \to f(a)$ for all sequences such that $a_n
  \to a$. Given $\epsilon > 0$, assume that instead there is no $\delta > 0$
  such that, for $d(a, y) < \delta$, $d(f(a), f(y)) <\epsilon$ for arbitrary $y
  \in M$. Then we can find $a_n \in M$ with $d(a, a_n) < 1/n$. However, this
  means $d(f(a), f(a_n)) \geq \epsilon$, which contradicts the assumption that
  $f(a_n) \to f(a)$ even though $a_n \to a$. So such $\delta$ exists and we have
  continuity. \qed
\end{proof}

\begin{lemma}
  The limit of a sequence is unique.
\end{lemma}

\begin{proof}
  Assume there are two limits $a$ and $b$ for the sequence $a_n$. Then $d(a,b)
  \leq d(a, a_n) + d(a_n, b)$. As $n\to\infty$, the RHS tends to zero so $a =
  b$. \qed
\end{proof}

A \textbf{Cauchy sequence} $(a_n)$ in the metric space $M$ is a sequence such
that, for all $\epsilon > 0$, there exists $n_0 \geq 0$ such that $d(a_p, a_q) <
\epsilon$ for all $p, q \geq n_0$.

\begin{lemma}
  A convergent sequence is a Cauchy sequence (the converse is not true).
\end{lemma}

\begin{proof}
  Suppose $a_n \to a$. Then, for all $\epsilon > 0$, there is some $n_0 \geq 0$
  such that $d(a_n, a) < \epsilon / 2$ for $n \geq n_0$. Let $p, q \geq n_0$,
  then $d(a_n, a_q) \leq d(a_p, a) + d(a_q, a) < \epsilon$, so the sequence is
  Cauchy. \qed
\end{proof}

A metric space $M$ is \textbf{complete} if all Cauchy sequences in $M$
converges.
\begin{theorem}
  The real line $\mathbb{R}$ is complete.
\end{theorem}

\begin{proof}
  Let $(a_n)$ be a Cauchy sequence in $\mathbb{R}$. Define the sequence of
  integers $(n_k)$ where $n_0 = 1$, and $n_{k+1}$ is the smallest integer bigger
  than $n_k$ where $| a_p - a_q | < 2^{-(k+2)}$ for $p, q \geq n_{k+1}$. Define
  the intervals $I_k = [a_{n_k} - 2^{-k}, a_{n_k} + 2^{-k}]$ and let $x \in
  I_{k+1}$. Now, since $x \in I_{k+1}$, this implies that $|x - a_{n_{k+1}}| <
  2^{-(k+1)}$. By definition of the integer sequence, $|a_{n_{k}} - a_{n_{k+1}}|
  < 2^{-(k+1)}$, so then, by triangle inequality,
  \begin{equation*}
    |a_{n_k} - x| \leq |x - a_{n_{k+1}}| + |a_{n_{k+1}} - a_{n_k}| < 
      2 \cdot 2^{-(k+1)} = 2^{-k},  
  \end{equation*}
  so $x \in I_k$. However, $x \in I_{k+1}$, so $I_{k+1} \subset I_k$. By
  Lemma~\ref{lem:intervals}, $\cap_{k=1}^\infty I_k \neq \emptyset$, so assume
  $a \in \cap_{k=1}^\infty I_k$. For $m \geq n_k$,
  \begin{equation*}
    |a - a_m| \leq |a - a_{n_k}| + |a_{n_k} - a_m| \leq 2^{-k} + 2^{-(k+1)}
    \to 0
  \end{equation*}
  as $m \geq n_k \to \infty$. Thus $a_m \to a$ and this arbitrary Cauchy
  sequence converges in $\mathbb{R}$ and thus $\mathbb{R}$ is complete. \qed
\end{proof}

\begin{proposition}
  For $X \neq \emptyset$, let $\mathcal{B}(X)$ be the set of functions $f : X
  \to \mathbb{R}$ such that $f$ is bounded. For $f, g, \in \mathcal{B}(X)$, let
  $d(f, g) = \sup_{x\in X} |f(x) - g(x)$. Then $(\mathcal{B}(X), d(f,g))$
  defines a complete metric space.
\end{proposition}

\begin{proof}
  $d$ is clearly a metric. For completeness, let $(f_n)$ be a Cauchy sequence in
  $\mathcal{B}(X)$. For $x\in X$, $(f_n(x))$ is a Cauchy sequence of real
  numbers because, by definition of $d(f,g)$, $|f_q(x) - f_p(x)| \leq d(f_p -
  f_q)$, and since $\mathcal{R}$ is complete, the sequence $(f_n(x))$ converges.
  
  Defining $f: X\to\mathbb{R}$ such that $f(x) = \lim_{n\to\infty} f_n(x)$, we
  need to show that $f\in\mathcal{B}(X)$, and that indeed $f_n(x) \to f(x)$
  regardless of $x\in X$. Be definition of a Cauchy sequence, for $\epsilon >0$,
  there exists $n_0 \geq 0$ such that $d(f_p, f_q) < \epsilon / 2$ for $p, q
  \geq n_0$. Note also that, for all $x\in X$, there exists $n_1(x) \geq n_0$
  such that $|f_{n_1(x)} - f| < \epsilon / 2$. Then, let $x_\in X$ and $n \geq
  n_0$, we have
  \begin{equation*}
    |f_n(x) - f(x)| \leq |f_n - f_{n_1(x)}| + |f_{n_1(x)} - f| < \epsilon.
  \end{equation*}
  Additionally note, $|f(x)| \leq |f(x) - f_{n_0(x)}| + |f_{n_0(x)}| \leq
  \epsilon + c_{f_{n_0}}$ since $f_{n_0(x)}$ is bounded, so $f \in
  \mathcal{B}(X)$. Further, $d(f_n - f) = \sup |f_n - f| = \delta < \epsilon$,
  so $f_n$ converges to $f \in \mathcal{B}(x)$. Thus every Cauchy sequence
  converges and thus the space is complete and equipped with a metric. \qed
\end{proof}

%-------------------------------------------------------------------------------

\section{Topology of metric spaces}

Let $(M, d)$ be a metric space with $\xb \in M$ and $r > 0$. Define the
\textbf{open ball} around $\xb$ of radius $r$ to be
\begin{equation*}
  B(\xb; r) = \{ \yb \in M\ |\ d(\xb, \yb) < r\}.
\end{equation*}
The analogous \textbf{closed ball} $D(\xb; r)$ is defined with the less than or
equal to sign. A set $A \subset M$ is \textbf{bounded} if it can be contained in
some $D(\xb; r)$ for some $\xb \in M$, $r > 0$. A set $U \subset M$ is
\textbf{open} if, for all $\xb \in U$, there exists $r_{\xb} > 0$ such that
$B(\xb; r_{\xb}) \subset U$. A set $A \subset M$ is \textbf{closed} if $M
\setminus A$ is open.

\begin{lemma}
Let $(M, d)$ be a metric space, then:
  \begin{enumerate}
    \item $M$ and $\emptyset$ are open;
    \item $\bigcup_i A_i$ is open if all $A_i \subset M$ are open;
    \item $\bigcap_i^n$ is open if all $A_i \subset M$ are open and $n <
    \infty$;
    \item $B(\xb; r)$ is open for some $r > 0$.
  \end{enumerate}
\end{lemma}

\begin{proof}
  The first two are obvious. For 3), suppose the open sets $U_i$ indexed by $i$
  are open and $\xb \in \bigcap_{i=1}^n U_i$. Then $\xb in U_i$ for all $i$, so
  there is some $B(\xb; r_i) \subset U_i$. Taking the minimum of such $r_i > 0$
  means $B(\xb; r_i) \subset \bigcap_{i=1}^n U_i$, and thus the collective
  finite union is open.
  
  For 4), let $\yb \in B(\xb; r)$, $r_y = r - d(\xb, \yb) > 0$ and $\zb \in
  B(\yb; r_y)$. Then $d(\xb, \zb) \leq d(\xb, \yb) + d(\yb, \zb) < d(\xb, \yb) +
  r - d(\xb, \yb) = r$, so $B(\yb; r_y) \subseteq B(\xb; r)$. \qed
\end{proof}

\begin{corollary}
The following may be shown by considering the appropriate complements:
  \begin{enumerate}
    \item $M$ and $\emptyset$ are closed;
    \item $\bigcap_i A_i$ is closed if $A_i \subset M$ for all $i$;
    \item $\bigcup_i A_i$ is closed if $A_i \subset M$ for all $i$ and $n <
    \infty$;
    \item $D(\xb; r)$ is closed.
  \end{enumerate}
  \qedwhite
\end{corollary}

\begin{example}
  Open intervals are open and closed intervals are closed.
  
  $(a, \infty)$ is open as it is a union of open bounded intervals.
  
  $[a, \infty)$ is closed since $(-\infty, a)$ is open.
  
  $\mathbb{Z}$ is closed as $\mathbb{R} \setminus \left(\cup_{n=-\infty}^\infty
  (n, n+1)\right)$ is closed.
  
  $\mathbb{Q}$ and $[0, 1)$ are neither, while $\mathbb{R}$ is both.
\end{example}

\begin{proposition}
  Suppose $M$ is a metric space and $A \subseteq M$. $A$ is closed iff every
  sequence converges to $a \in A$.
\end{proposition}

\begin{proof}
  Assume $A$ is closed and $a_n \to a$. Assume the converse so that $a \in U = M
  \setminus A$ which is an open set. Then there is some $r > 0$ such that $B(a;
  r) \in U$, and since $a_n \to a$, there exists $n_0 \geq 0$ where $d(a_n, a) <
  r$ for $n \geq n_0$. This implies $a_n \in B(a; r)$ for all $n$, but this is a
  contradiction since $a_n \in A$, and thus $a \in A$.
  
  Assume $a_n \to a \in A$. Let $x \in M\setminus A$, $r > 0$, and assume there
  is no such $B(x; r) \subset M\setminus A$. Thus there is an intersection,
  i.e., $B(x; 1/n) \cap A \neq \emptyset$. This implies that there is some $i$
  where $a_i \in B(x; 1/n) \cap A$. However, $(a_n)$ is a sequence in $A$ and
  $d(a_m, x) < 1/n$ for $m \geq n+1$, so $a_m \to a$, but this implies $x = a$
  which is not possible since $x \in M\setminus A$. So $M\setminus A$ is open
  which means $A$ is closed. \qed
\end{proof}

\begin{theorem}
  Let $M$ be a complete metric space and $A\subseteq M$ is closed. Then $A$ is
  complete with the induced metric.
\end{theorem}

\begin{proof}
  Let $(a_n)$ be a Cauchy sequence in $A$. Since $M$ is complete, $(a_n)$
  converges in $M$, but $A$ is closed, so $(a_n)$ converges in $A$ by previous
  proposition, which implies $A$ is complete. \qed
\end{proof}

Let $M$ be a metric space. $M$ is \textbf{compact} if every sequence $(a_n) \in
M$ has a convergent subsequence $(a_{n_k})$.
\begin{example}
  \begin{itemize}
    \item $(a_n) = (-1)^n$ is non-convergent but has a convergent sequence.
    
    \item $M = (0, 1)$ is not compact since $a_n = 1/n$ and its subsequences do
    not converge in $M$.
    
    \item $\mathbb{R}$ is not compact as $a_n$ has no subsequence converging in
    $\mathbb{R}$.
    
    \item $M = [0, 1]$ is compact. Let $(a_n)$ be a subsequence in $M$. Let
    $I_1$ be either $[0, 1/2]$ or $[1/2, 1]$, and let $(a_{n_k})$ be the
    subsequences in $I_1$. Continuing this we have a sequence of intervals
    $I_{m+1} \subset I_m$ with $I_m$ of length $2^{-m}$. Denote the subsequences
    $(a_{m_k}^m)$ to be those in $I_m$. Taking $b_m = a_{n_m}^m \in I_m$, we see
    that $b_{m+1} \in I_M$ since $I_{m+1} \subset I_m$, so that $d(b_m, b_q)
    \leq 2^{-m}$ for $q \geq m$. Thus $(b_m)$ is a Cauchy sequence, which is a
    subsequence of $(a_n)$. Since $M \subseteq \mathbb{R}$, $M$ is complete, so
    $b_m \to b \in M$, and thus $M$ is compact.
  \end{itemize}
\end{example}

\begin{proposition}
  By extension, closed $n$-gons in $\mathbb{R}^n$ are compact. \qedwhite
\end{proposition}

\begin{proposition}
  Let $f : M \to N$ be a continuous map between metric spaces. If $M$ is
  compact, then $f(M) \subset N$ is compact.
\end{proposition}

\begin{proof}
  Let $(a_n)$ be a sequence in $f(M)$. Then $a_n = f(b_n)$ for some $b_n \in M$.
  The sequence $(b_{n_k})$ converges in $M$ since $M$ is compact, thus
  \begin{equation*}
    \lim_{k\to\infty} a_{n_k} = \lim_{k\to\infty} f(b_{n_k})
    = f \left(\lim_{k\to\infty} b_{n_k} \right) = f(b)
  \end{equation*}
  since $f$ is continuous. So $(a_{n_k})$ is convergent, thus $f(M)$ is compact.
  \qedwhite
\end{proof}

\begin{proposition}
  A closed subset of a compact space is a compact set.
\end{proposition}

\begin{proof}
  Let $(a_n)$ be a sequence in $A \subset M$ where $M$ is compact. Since $(a_n)
  \in M$, $(a_{n_k})$ is convergent, but $A$ closed so $(a_{n_k}) \to a \in A$,
  thus $A$ is compact. \qedwhite
\end{proof}

\subsection{Heine--Borel theorem}

\begin{theorem}
  A subset $A \subseteq \mathbb{R}^n$ is compact iff $A$ is closed and bounded.
\end{theorem}

\begin{proof}
  Suppose $A$ is compact, so clearly $A$ is closed. If $A$ is unbounded, then
  there exists $(a_n) \in A$ where $d(a_n, 0) \geq n$, so $(a_{n_k})$ does not
  converge in $\mathbb{R}^n$. However $A$ is compact, which is a contradiction,
  so $A$ is bounded.
  
  Suppose $A$ is bounded, then $A\subseteq [a,b]^n$. If $A$ is closed, then it
  is a closed subset of a compact set, so $A$ is compact by previous
  proposition. \qed
\end{proof}

For example, if $f: M \to N$ with $f$ is a scalar continuous function, then
$f(M) \subset \mathbb{R}$ is closed and bounded since $M$ is compact, and thus
$f(M)$ compact implies $f(M)$ is closed and bounded.

%-------------------------------------------------------------------------------

\section{Banach and Hilbert spaces}

Let $V$ be a real vector space. The \textbf{norm} on $V$ is a function
$\|\cdot\| : V \to [0, \infty)$ where:
\begin{enumerate}
  \item $\|\xb\| = 0$ iff $\xb = \boldsymbol{0}$;
  \item $\|\lambda \xb \| = |\lambda| \cdot \|\xb\|$ for all $\xb \in V$ and
  $\lambda \in \mathbb{R}$;
  \item $\|\xb + \yb\| \leq \|\xb\| + \|\yb\|$.
\end{enumerate}
The pair $\left(V, \| \cdot \|\right)$ gives a \textbf{normed vector space}.

\begin{lemma}
  Let $V$ be a normed vector space, then $d(\xb, \yb) = \|\xb - \yb\|$ defines a
  metric on $V$.
\end{lemma}

\begin{proof}
  Two of the properties follow from definition. To show the reflexive property,
  note that
  \begin{equation*}
    d(\yb, \xb) = \| \yb - \xb \| = \| (-1)(\xb - \yb)\| = 
    \| \xb - \yb \| = d(\xb, \yb).
  \end{equation*}
  \qed
\end{proof}

\begin{example}
  \begin{enumerate}
    \item It may be shown that the metrics
    \begin{equation*}
      \sum_i |x_i|, \qquad \sum_i \sqrt{|x_i|^2}, \qquad
      \max \{|x_i| \in \mathbb{R}\}
    \end{equation*}
    define norms on $\mathbb{R}^n$ (the $\ell^1$, $\ell^2$ and $\ell^\infty$
    norms).
    
    \item The \textbf{supremum norm} on $B(X)$ is defined by
    \begin{equation*}
      \|f\|_{\infty} = \sup\{ |f(x)|\in \mathbb{R}\ ;\ x\in X\}.
    \end{equation*}
    
    \item For $X$ a metric space, $C_b (X) = \{f : x \to \mathbb{R}\ |\ f\
    \textnormal{continuous and bounded}\}$ is also a normed vector space with
    the supremum norm.
  \end{enumerate}
\end{example}

If $C(X) = \{f : x \to \mathbb{R}\ |\ f\ \textnormal{continuous}\}$ then $f$
does not have a supremum, however, we have the following:
\begin{proposition}
  If $X$ is compact, then $C(X) = C_b(X)$, so $C(X)$ is a normed vector space.
\end{proposition}

\begin{proof}
  $C_b(X) \subseteq C(X)$ regardless of $X$. For the converse, assume $f \in
  C(X)$, so that $f(X)$ is compact. This implies $f(X)$ is bounded and closed by
  the Heine--Borel theorem, so $C(X) \subseteq C_b(X)$. \qed
\end{proof}

Let $(V, \|\cdot\|_V)$ and $(W, \|\cdot\|_W)$ be two normed vector spaces. A
function $f : V \to W$ is continuous at $\xb \in V$ if, for all $\epsilon > 0$,
there exists $\delta > 0$ such that $\| \xb - \yb\|_V < \delta$ implies that
$\|f(\xb) - f(\yb)\|_W < \epsilon$.

Let $V$ be a normed vector space. $V$ is a \textbf{Banach space} if $V$ with
the metric induced by the norm is complete.

\begin{theorem}
  Let $X$ be a metric space, then $C_b(X)$ with the supremum norm is a Banach
  space.
\end{theorem}

\begin{proof}
  Since $C_b(X) \subseteq B(X)$, if $C_b$ is closed, then $C_b$ is complete
  since $B(X)$ is complete. To show this, let $(f_n) \in C_b(X)$, and let $f_n
  \to f \in B(X)$. The convergene of $f_n$ implies that there exists $n_0 \geq
  0$ such that $\|f_n - f\| < \epsilon /3$ for any $\epsilon > 0$ with $n \geq
  n_0$. Also, $\|f_{n_0}(y) - f(y)\| < \epsilon /3$ for all $y \in X$. The
  functions are continuous, so there exists $\delta > 0$ where, if $d(x, y) <
  \delta$, $\|f_{n_0}(x) - f_{n_0}(y)\| < \epsilon /3$ for $x \in X$. Thus, for
  $d(x, y) < \delta$,
  \begin{equation*}
    |f(x) - f(y)| \leq |f(x) - f_{n_0}(x)| + |f_{n_0}(x) - f_{n_0}(y)|
      + |f_{n_0}(y) - f(y)| < \epsilon,
  \end{equation*}
  so $f$ is continuous, and $C_b(X)$ is closed and thus complete. \qed
\end{proof}

\begin{corollary}
  For $a < b$, $C[a,b]$ with the supremum norm is a Banach space. \qedwhite
\end{corollary}

Note that $C[a,b]$ is not a complete space with, for example, the $L_2$ norm
\begin{equation*}
  \| f \|_2 = \sqrt{\int_a^b \left(f(x)\right)^2\, \mathrm{d}x}.
\end{equation*}
For example, with $f_n = x^n$, $f_n \to 0$ but clearly $f_n(1) = 1$ for all $n$.
The underlying reason is the sequence is not a Cauchy sequence with respect to
the norm.

Convergence with respect to the supremum norm is called \textbf{uniform
convergence} (cf. Complex Analysis 2H).

Let $(V, \|\cdot\|)$ be a Banach space. If there is an inner product from $V$
which induces this norm, then $V$ is called a \textbf{Hilbert space}.

\begin{theorem}
  Let $(M, d)$ be a metric space. Then there exists $(\overline{M},
  \overline{d})$ where $\overline{M}$ is complete, and there is an embedding
  $\iota : M \to \overline{M}$ with $d(x,y) = d(\iota(x), \iota(y))$ for all $x,
  y \in M$. Also, for all $\overline{x} \in \overline{M}$, there is a sequence
  $(x_n) \in M$ with $x_n \to \overline{x}$ as $n \to \infty$. \qedwhite
\end{theorem}

Here, $\overline{M}$ is called the \textbf{completion} of $M$, and it is unique
up to some isomorphism.

\begin{example}
  The completion of $\mathbb{Q}$ is $\mathbb{R}$ with respect to the Euclidean
  metric.
  
  The completeness of $C[a,b]$ with respect to the inner product metric is
  denoted $L^2[a,b]$.\marginnote{Note that elements of $L^2$ are not exactly
  functions, but rather \emph{equivalence classes} (cf. $11 \equiv 1$ modulo
  10)}.
\end{example}

%%%%%%%%%%%%%%%%%%%%%%%%%%

\subsection{The contraction mapping theorem}

\begin{theorem}
  Let $(M,d)$ be a complete metric space, $0\leq\lambda\leq1$ and a $f: M\to M$
  with $d(f(x), f(y)) \leq \lambda d(x,y)$ for all $x,y\in M$. Then $f$ has one
  unique fixed point where $f(x_0) = x_0$.\marginnote{Or, if you throw a map of
  the world on the floor, there is exactly one point on the map that exactly
  corresponds to one point on the floor.}
\end{theorem}

\begin{proof}
  Note that $f$ is a contraction, and continuity is automatically satisfied from
  the condition that $d(f(x), f(y)) \leq \lambda d(x,y)$.
  
  Let $x\in M$, and $a_n = f^n(x)$. So we have
  \begin{align*}
    d(x, a_n) &\leq d(x, f(x)) + d(f(x), f^2(x)) + \ldots d(f^{n-1}(x), f^n(x)) \\
      &= \sum_{i=0}^{n-1} d(f^i(x), f^{i+1}(x))\\
      &\leq \sum_{i=0}^{n-1} \lambda d(x, f(x))\\
      &= d(x, f(x)) \frac{1-\lambda^n}{1 - \lambda}\\
      &\leq \frac{d(x, f(x))}{1 - \lambda},
  \end{align*}
  by Cauchy--Schwartz and the arithmetic progression with $0\leq\lambda< 1$.
  Now,
  \begin{equation*}
    d(a_n, a_m) = d(f^n(x), f^m(x)) \leq \lambda^m d(f^{n-m}, x) \leq \lambda^m \frac{d(x, f(x))}{1-\lambda}
  \end{equation*}
  assuming $n>m$. For $n,m \geq n_0$, we have
  \begin{equation*}
    d(a_n, a_m) \leq \lambda^{n_0} \frac{d(x, f(x))}{1-\lambda}.
  \end{equation*}
  Clearly $(a_n)$ is a Cauchy sequence, and thus we have completeness and $a_n
  \to a \in M$. Now,
  \begin{equation*}
    f(a) = f\left(\lim_{n\to\infty} a_n\right) = \lim_{n\to\infty} f(a_n) = \lim_{n\to\infty} a_{n+1} = a,
  \end{equation*}
  so there is some $a\in M$ that is a fixed point.
  
  To show uniqueness, suppose $b$ is another fixed point. Then
  \begin{equation*}
    d(a,b) = d(f(a), f(b)) \leq \lambda d(a,b),
  \end{equation*}
  and for $\lambda \neq 0$, $d(a,b)=0$, so $a=b$. \qed
\end{proof}

%-------------------------------------------------------------------------------

\section{A norm for matrix spaces}

We want a norm reflecting the fact that matrices can be identified with linear
maps. Let $\boldsymbol{A} = (A_{ij}) \in M_{n,k}(\mathbb{R})$. We define
\begin{equation}
  \| \boldsymbol{A} \| = \sup \{ \|\boldsymbol{A}\boldsymbol{x}\|_2 \ : \ \boldsymbol{x} \in \mathbb{R}^k,\ \|\boldsymbol{x}\|_2 \leq 1\},
\end{equation}
where $\|\cdot\|$ is the Euclidean norm. Here, $\boldsymbol{A}\boldsymbol{x} \in
\mathbb{R}^n$, and $\boldsymbol{x} \mapsto \|\boldsymbol{A}\boldsymbol{x}\|_2$
is clearly a continuous map. By the Heine--Borel theorem, $\{
\|\boldsymbol{A}\boldsymbol{x}\|_2 : \|\boldsymbol{x}\|_2 \leq 1\}$ is bounded
and closed, so the supremum exists, and there is $\boldsymbol{x}$ with
$\|\boldsymbol{x}\|_2 \leq 1$ such that $\| \boldsymbol{A} \| =
\|\boldsymbol{A}\boldsymbol{x}\|_2$ exists.

\begin{lemma}
  We have
  \begin{itemize}
    \item $\|\boldsymbol{A}\boldsymbol{x}\|_2 \leq \|\boldsymbol{A}\|
    \|\boldsymbol{x}\|_2$ for all $\boldsymbol{A}$ and $\boldsymbol{x}$

    \item $\|\boldsymbol{A}\boldsymbol{\mathsf{B}}\| \leq \|\boldsymbol{A}\|
    \|\boldsymbol{\mathsf{B}}\|$
    
    \item $\|\boldsymbol{A}\|_\infty \leq \|\boldsymbol{A}\| \leq k\sqrt{n}
    \|\boldsymbol{A}\|_\infty$,
  \end{itemize}
  where $\|\boldsymbol{A}\|_\infty = \max \{ |A_{ij}|\ : \ \boldsymbol{A} \in
  M_{n,k}(\mathbb{R})\}$. \qedwhite
\end{lemma}

%===============================================================================

\chapter{Ordinary differential equations}

Let $U \subset \mathbb{R}^n$ be open. A \textbf{vector field} or
\textbf{autonomous differential equation} is a continuous map $v : U \to
\mathbb{R}^n$ with no explicit time dependence. Here, $U$ is called the
\textbf{phase space} of $v$.

For $\boldsymbol{x} \in U$, $\tau \in \mathbb{R}$, a continuous differential
curve $\alpha : (a,b) \to U$ is an \textbf{integral curve} of $v$ at
$(\boldsymbol{x}, \tau)$ if $\tau \in (a,b)$, $\alpha(t) = \boldsymbol{x}$ and
$\alpha'(t) = v(\alpha(t))$. Note the integral curves have tangent vectors which
agree with $v$ at a given point.

More generally, for $U \in \mathbb{R}^n$, $I \subset \mathbb{R}$, a
\textbf{differential equation} is a continuous map $V : U\times I \to
\mathbb{R}^n$. A \textbf{solution} of $V$ at $\boldsymbol{x} \in U$ and $\tau
\in I$ is a continuously differential curve $\alpha : I \to U$ with $\alpha'(t)
= V(\alpha(t), t)$ and $\alpha(t) = \boldsymbol{x}$.

%-------------------------------------------------------------------------------

\section{Picard--Lindel\"of theorem}

This is an existence and uniqueness theorem for differential equations.

\begin{theorem}
  Let $U \subset \mathbb{R}^n$, $I \subset \mathbb{R}$ be open and $V : U\times
  I \to \mathbb{R}^n$ be a differential equation where, for all
  $\boldsymbol{x}_1, \boldsymbol{x}_2 \in U$, $t \in I$, there exists $L \geq 0$
  such that\marginnote{Compare this with the \textbf{Lipschitz condition} where
  $\|v(\boldsymbol{x}_1) - v(\boldsymbol{x}_2)\| \leq L \|\boldsymbol{x}_1 -
  \boldsymbol{x}_2\|$, where $L$ is the \textbf{Lipschitz constant}.}
  \begin{equation*}
    \|v(\boldsymbol{x}_1, t) - v(\boldsymbol{x}_2, t)\| \leq L \|\boldsymbol{x}_1 - \boldsymbol{x}_2\|_2.
  \end{equation*}
  Given $(\boldsymbol{u}, \tau) \in U \times I$, there exists $a,b>0$ with
  \begin{equation*}
    U_1 = \{\boldsymbol{x} \in U\ :\ \|\boldsymbol{x} - \boldsymbol{u}\| < a\}, \quad
    I_1 = \{t \in I\ :\ |t - \tau| < b\}
  \end{equation*}
  such that the differential equation $v$ has an unique solution for all
  $\boldsymbol{x} \in U_1$ and $\tau \in I_1$. Furthermore, the resulting
  $\alpha : U_1 \times I_1 \to U$ given by $\alpha(\boldsymbol{x}, t) =
  \alpha_{\boldsymbol{x}}(t)$ is continuous.
\end{theorem}

\begin{proof}
  This one is quite long! The key idea is to construct a contraction mapping $A$
  and make use of the fixed point theorem to demonstrate existence and
  uniqueness. We are going to split this up into little bits.
  
  \begin{itemize}
    \item We first construct an integral curve $\alpha$ with
    $\partial\alpha/\partial t (\boldsymbol{x}, t) = v(\alpha, t)$,
    $\alpha(\boldsymbol{x}, \tau) = \boldsymbol{x}$. By integrating,
    \begin{equation*}
      \alpha(\boldsymbol{x}, t) = \boldsymbol{x} + \int_\tau^t v(\alpha(\boldsymbol{x}, s), s)\; \mathrm{d}s.
    \end{equation*}
    Define some operator $A$ such that
    \begin{equation*}
      A\beta(\boldsymbol{x}, t) = \boldsymbol{x} + \int_\tau^t v(\beta(\boldsymbol{x}, s), s)\; \mathrm{d}s,
    \end{equation*}
    then we note that $A\alpha =\alpha$, and $\alpha$ is a fixed point of the
    operator $A$. We aim to show that $A$ is a contraction in a space satisfying
    the relevant properties.
    
    \item Let $a_1, b_1 > 0$ be such that
    \begin{equation*}
      D_1 = D(\boldsymbol{u}; 2a_1) \subset U, \quad D_2 = D(\tau; b_1) \subset I.
    \end{equation*}
    By the Heine--Borel theorem\marginnote{Recall $D$ denotes \emph{closed}
    balls, while $B$ denote open balls.}, $D_1 \times D_2 \subset
    \mathbb{R}^{n+1}$ is compact, and so there exists some $K \geq 0$ such that,
    with respect to the Euclidean norm, $\|\boldsymbol{v}(\boldsymbol{x}, t)\| <
    K$ for all $(\boldsymbol{x}, t) \in D_1 \times D_2$.
    
    Let $a, b > 0$ be such that
    \begin{equation*}
      0 < a < a_1, \qquad b < \min \left\{b_1, \frac{a}{K}, \frac{1}{L}\right\}.
    \end{equation*}
    Recall that $U_1 = B(\boldsymbol{u}; a)$ and $I_1 = B(\tau; b)$, so let
    \begin{equation*}
      M = \{\beta\ :\ U_1 \times I_1 \to D \subset \mathbb{R}^n \}
    \end{equation*}
    where $\beta$ is continuous and $\beta(\boldsymbol{x}, \tau) =
    \boldsymbol{x}$ for all $\boldsymbol{x}\in U_1$. This implies that
    \begin{equation*}
      M \subseteq \left( C_b(U_1 \times I_1) \right)^n,
    \end{equation*}
    and since $\left( C_b(U_1 \times I_1) \right)^n$ is a Banach space with the
    supremum norm, if $M$ is closed, then $M$ is complete.
    
    \item Suppose $(\beta_n) \in M$ where $\beta_n \to \beta$. For
    $(\boldsymbol{x}, t) \in U_1 \times I_1$, $\|\beta(\boldsymbol{x}, t) -
    \beta_n (\boldsymbol{x}, t)\| \leq \|\beta - \beta_n\|$ so $\beta_n \to
    \beta$, but since $D_1$ is closed, $\beta \in D$ and obviously
    $\beta_n((\boldsymbol{x}, \tau) \to \beta(\boldsymbol{x}, \tau) =
    \boldsymbol{x}$, so $M$ is closed and so is complete.
    
    \item If we now consider $A\beta$, then we have $A\beta(\boldsymbol{x},
    \tau) = \boldsymbol{x}$ and that
    \begin{align*}
      \|A\beta(\boldsymbol{x}, t) - \boldsymbol{u}\| & \leq \|A((\boldsymbol{x}, t) - \boldsymbol{x}\|  - \| \boldsymbol{x} - \boldsymbol{u}\|\\
        &\leq \int_\tau^t \|v(\beta(\boldsymbol{x}, s), s\|\; \mathrm{d}s + a\\
        &\leq K|t - \tau| + a\\
        &\leq Kb + a\\
        &< 2a < 2a_1 
    \end{align*}
    by Cauchy--Schwartz, definition of $U_1$, second bullet point, definition of
    $I_1$, and definition of $b$ and $a$ respectively. By definition of $D_1$,
    we have $A\beta(\boldsymbol{x}, t) \in D_1$.
    
    \item Note then we have
    \begin{align*}
      \|A\beta(\boldsymbol{x}, t) - A\beta(\boldsymbol{y}, t') &\leq \|\boldsymbol{x} - \boldsymbol{y}\| + \left\|\int_\tau^t v(\beta(\boldsymbol{x}, s), s) - v(\beta(\boldsymbol{y}, s), s)\; \mathrm{d}s\right\|\\
        &\quad + \left\|\int_t^{t'} v(\beta(\boldsymbol{y}, s), s)\; \mathrm{d}s\right\|\\
      &\leq \|\boldsymbol{x} - \boldsymbol{y}\| + L\int_\tau^t \left\|v(\beta(\boldsymbol{x}, s), s) - v(\beta(\boldsymbol{y}, s), s)\right\|\; \mathrm{d}s\\
        &\quad + K|t - t'|\\
      &\leq \|\boldsymbol{x} - \boldsymbol{y}\| + L \sup_{s\in[\tau, t]}\|\beta(\boldsymbol{x}, s) - \beta(\boldsymbol{y}, s)\| + K|t - t'|,
    \end{align*}
    by the Lipshitz conditions. All terms can be made arbitrarily small since
    $\boldsymbol{x}$ can be made close to $\boldsymbol{y}$, $t$ can be made
    close to $t'$, and since $[\tau, t]$ is compact, $\|\beta(\boldsymbol{x}, s)
    - \beta(\boldsymbol{y}, s)\|$ can be made arbitrarily small. So now $A\beta
    \in D_1$ is continuous, and therefore $A\beta \in M$, and $A : M \to M$ is a
    self mapping.
    
    \item Since $A$ is a self-mapping, for $\beta_{1,2} \in M$, we have
    \begin{align*}
      \|A\beta_1 - A\beta_2\| &\leq \int_\tau^t \|v(\beta_1(\boldsymbol{x}, s), s) - v(\beta_2(\boldsymbol{x}, s), s) \|\; \mathrm{d}s \\
        &\leq L \int_\tau^t \|\beta_1 - \beta_2\|\; \mathrm{d}s\\
        &= L |t - \tau| \|\beta_1 - \beta_2\|\\
        &\leq (Lb) \|\beta_1 - \beta_2\|
    \end{align*}
    by definition of $I_1$. Note that $Lb < 1$ by the definition of
    $b$\marginnote{$b < 1 / L$.}, and therefore $A$ is a contraction.
  \end{itemize}
  Since $A$ is a contraction and $M$ is complete, by contraction mapping there
  is one unique point in $M$ that is fixed under $A$. Clearly this is $\alpha$
  by definition of $\beta$ (see first bullet point), and hence $\alpha$ is the
  unique solution to the ODE satisfying the stated conditions. \qed
\end{proof}

Note that it doesn't matter if $\alpha : I_1 \to U$, since we can redefine $M$
and $A$ as $M_{\boldsymbol{x}} = \{\beta\ :\ I_1 \to D\}$ with $\beta(t) =
\boldsymbol{x}$, and $A_{\boldsymbol{x}} : M_{\boldsymbol{x}} \to
M_{\boldsymbol{x}}$. There will be an unique solution for fixed $\boldsymbol{x}
\in U_1$, where the generation solution gives this solution.

%-------------------------------------------------------------------------------

\section{Differentiation in $\mathbb{R}^n$}

Let $U \subset \mathbb{R}^n$ be open. Recall that $f : U \to \mathbb{R}^n$ is differentiable at $\boldsymbol{x} \in U$ with derivative
\begin{equation}
  Df(\boldsymbol{x}) = \left(\frac{\partial f_i}{\partial x_j}\right) \in M_{p,n}(\mathbb{R})
\end{equation}
if near $\boldsymbol{x}$ we can write
\begin{equation*}
  f(\boldsymbol{x} + \boldsymbol{h}) = f(\boldsymbol{x}) + Df(\boldsymbol{x}) \cdot \boldsymbol{h} + R(\boldsymbol{h}), \qquad \lim_{\|\boldsymbol{h}\|\to 0} \frac{R(\boldsymbol{h})}{\|\boldsymbol{h}\|} = 0.
\end{equation*}
If $f$ is differential for all $\boldsymbol{x} \in U$, then $Df : U \to M_{p,
n}(\mathbb{R}) = \mathbb{R}^{pn}$. If $D^i f$ is continuous then $f$ is said to
be of \textbf{$i$-class}, with $f \in C^i (U)$.

%%%%%%%%%%%%%%%%%%%%%%%%%%%%%%

\subsection{Mean value theorem}

\begin{theorem}
  Let $U \subset \mathbb{R}^n$ be open, $\boldsymbol{x} \in U$, $\boldsymbol{h}
  \in \mathbb{R}^n$ where $\boldsymbol{x} + t\boldsymbol{h} \in U$ for all $t
  \in [0, 1]$ and $f \in C^1 : U \to \mathbb{R}^p$, then
  \begin{equation*}
    f(\boldsymbol{x} + \boldsymbol{h}) - f(\boldsymbol{x}) = \int^1_0 Df(\boldsymbol{x} + t\boldsymbol{h})\cdot\boldsymbol{h}\; \mathrm{d}t.
  \end{equation*}
\end{theorem}

\begin{proof}
  Let $f_i : U \to \mathbb{R}$ with $g_i(t) = f_i(\boldsymbol{x} +
  t\boldsymbol{h})$, so that $g : [0, 1] \to \mathbb{R}$. Then we have $g_i'(t)
  = Df_i(\boldsymbol{x} + t\boldsymbol{h}) \cdot \boldsymbol{h}$. By the
  fundamental theorem of calculus,
  \begin{align*}
    g_i(1) - g_i(0) &= \int_0^1 Df_i(\boldsymbol{x} + t\boldsymbol{h}) \cdot \boldsymbol{h}\; \mathrm{d}t \\
      &= f_i(\boldsymbol{x} + t\boldsymbol{h}) - f_i(\boldsymbol{x}).
  \end{align*}
  Since this is true per component, we have the result in higher dimensions. \qed
\end{proof}

\begin{corollary}
  Let $U \subset \mathbb{R}^n$ be open and convex\sidenote{So for all
  $\boldsymbol{x}, \boldsymbol{y} \in U$, $\boldsymbol{x} t +
  (1-t)\boldsymbol{y} \in U$ for $t\in[0, 1]$.}, and also that $f \in C^1 : U
  \to \mathbb{R}^n$. Assume that there exists some $C = \sup
  \{\|Df(\boldsymbol{x})\| \in \mathbb{R}\ :\ \boldsymbol{x} \in U\}$, then
  $\|f(\boldsymbol{y}) - f(\boldsymbol{x})\| \leq C\|\boldsymbol{y} -
  \boldsymbol{x}\|$.
\end{corollary}

\begin{proof}
  By the mean value theorem, we have
  \begin{align*}
    \|f(\boldsymbol{x} + \boldsymbol{h}) - f(\boldsymbol{x})\| &\leq \int_0^1 \|Df(\boldsymbol{x} + \boldsymbol{h} \cdot \boldsymbol{h}\|\; \mathrm{d}t\\
      &\leq \int_0^1 \|Df(\boldsymbol{x} + \boldsymbol{h})\| \cdot \|\boldsymbol{h}\|\; \mathrm{d}t\\
      &\leq \int_0^1 C \cdot \|\boldsymbol{h}\|\; \mathrm{d}t = C \cdot \|\boldsymbol{h}\|.
  \end{align*}
  Since $\boldsymbol{h}$ is arbitrary (up to us assuming convexity), letting
  $\boldsymbol{h} = \boldsymbol{y} - \boldsymbol{x}$ leads the result. \qed
\end{proof}

Note that for the above corollary, $U$ can always be reduced so that $C$ exists
locally. For the Picard--Lindel\"of theorem, we get $v \in C^1 : U \times I \to
\mathbb{R}$ implies the Lipschitz condition is satisfied locally.

%%%%%%%%%%%%%%%%%%%%%%%%%%%%%%

\subsection{Matrices}

Let $U \subset \mathbb{R}^n$ and $V \subset \mathbb{R}^p$ be open. A
$C^1$-function $f : U \to V$ is a \textbf{diffeomorphism} if there exists
$f^{-1} : V \to U$ where $f \circ f^{-1} = f^{-1} \circ f = \mbox{id}$ (the
identity map), and $f^{-1}$ is differential for all $\boldsymbol{x} \in V$.

\begin{example}
  $f = x^3$ has $f^{-1} = x^{1/3}$, but since $f^{-1}$ is not differentiable at
  $x=0$, $x^3$ is not a diffeomorphism on $\mathbb{R}$.
\end{example}

By the chain rule, note that
\begin{equation*}
  D(f^{-1} \circ f) = \left(Df^{-1}(f)\right) Df = I_n, \quad D(f \circ f^{-1}) = \left(Df(f^{-1})\right) Df^{-1} = I_p.
\end{equation*}
If $\boldsymbol{y} = f(\boldsymbol{x})$ then $Df^{-1}(\boldsymbol{y}) =
(Df(\boldsymbol{x}))^{-1}$, then inverse matrix of $Df(\boldsymbol{x})$, so
$Df(\boldsymbol{x})$ is invertible and $p=n$ if $f$ is a diffeomorphism.

\begin{lemma}
  \begin{enumerate}
    \item $\mbox{GL}_n(\mathbb{R})$ is an open set\marginnote{This is the
    general linear group with real entries.}.
    \item $A \in M_{n,n}(\mathbb{R})$ with $\|A\| \leq 1$ implies that $I - A \in
    \mbox{GL}_n(\mathbb{R})$.
    \item $\mbox{inv} : \mbox{GL}_n(\mathbb{R}) \to \mbox{GL}_n(\mathbb{R})$
    with $A \mapsto A^{-1}$ is a smooth diffeomorphism.
  \end{enumerate}
\end{lemma}

\begin{proof}
  Recall that the determinant is defined as\marginnote{$S_n$ here is the group
  of symmetric permutations, and $\mbox{sig}(\sigma)$ is the signature of the
  permutation $\sigma$ ($+1$ if even and $-1$ if odd).}
  \begin{equation*}
    \mbox{det}\ A = |A| = \sum_{\mbox{sig} \in S_n} \mbox{sign}(\sigma) \prod_{i=1}^n a_{i, \sigma(i)}.
  \end{equation*}
  This is a polynomial in components of $A$, so it is a smooth function.
  
  \begin{enumerate}
    \item $\mbox{GL}_n(\mathbb{R}) = \mbox{det}^{-1}(\mathbb{R} - \{0\})$ so $A
    \in \mbox{GL}_n(\mathbb{R})$ implies that $|A| \neq 0$, which implies $|B|
    \neq 0$ for $B$ close to $A$, and thus $\mbox{GL}_n(\mathbb{R})$ is open.
    
    \item If $\|A\| \leq 1$, define $B_n = \prod_{i=0}^n A^i$ where $A^0 = I$.
    $\{B_n\}$ is a Cauchy sequence since
    \begin{equation*}
      \|B_n - B_m\| \leq \sum_{k = \min\{m,n\} + 1}^{\max\{m,n\}} \|A\|^k \leq \frac{\|A\|}{1 - \|A\|} \to 0
    \end{equation*}
    for sufficiently large $m,n$ with $\|A\| \leq 1$. So there exists $B =
    \lim_{n\to\infty} B_n$, and thus
    \begin{equation*}
      (I - A) B = (I - A) \lim_{n\to\infty} B_n.
    \end{equation*}
    $B_n$ continuous implies that
    \begin{equation*}
      (I - A)B = \lim_{n\to\infty} (I - A) B_n = \lim_{n\to\infty} I - A^{n+1} = I
    \end{equation*}
    since $\|A\| \leq 1$, so $B^{-1} = I - A \in \mbox{GL}_n(\mathbb{R})$.
    
    \item By Cramer's rule, for $A = (a_{ij})$, $A^{-1} = (b_{ij})$ with $b_{ij}
    = \mbox{det} A_{ij} / \mbox{det} A$, where $A_{ij}$ is the matrix obtained
    by replacing the $i^{\textnormal{th}}$ column with the standard
    $j^{\textnormal{th}}$ basis vector. So $(b_{ij})$ depends smoothly on
    $(a_{ij})$ since $\mbox{det}$ is a smooth map, and so $\mbox{inv}$ is
    smooth. Note additionally that $\mbox{inv}\circ\mbox{inv} = \mbox{id}$, so
    it is a bijection and hence a diffeomorphism.
  \end{enumerate}
  \qed
\end{proof}

%%%%%%%%%%%%%%%%%%%%%%%%%%%%%%

\subsection{Inverse function theorem}

Let $U \subset \mathbb{R}^n$ be open and $f \in C^k : U \to \mathbb{R}^n$. $f$
is \textbf{locally invertible} at $\boldsymbol{x} \in U$ if there exists $U_1
\subset U$ such that for $\boldsymbol{x} \in U_1$, $V_1 \subset \mathbb{R}^n$
where $f(\boldsymbol{x}) \in V_1$ is open and $f: U_1 \to V_1$ is a
diffeomorphism.

\begin{theorem}
  Let $U \subset \mathbb{R}^n$ be open and $f \in C^k : U \to \mathbb{R}^n$,
  $\boldsymbol{u} \in U$. $f$ is locally invertible iff $Df(\boldsymbol{u})$ is
  invertible. Here the local inverse is of class $C^k$.
\end{theorem}

\begin{proof}
  This one is quite long!
  
  \begin{itemize}
    \item If $f$ is locally invertible at $\boldsymbol{u}$, then it is a
    diffeomorphism, so clearly $Df(\boldsymbol{u})$ is invertible. However, this
    is for an isolated point, and we need to show that is also true on the
    appropriate neighbourhood.
    
    \item Assume that $\boldsymbol{u} = \boldsymbol{0} = f(\boldsymbol{u})$,
    i.e. a fixed point, and $Df(\boldsymbol{0}) = I$. Define, for
    $\boldsymbol{y} \in \mathbb{R}^n$,
    \begin{equation*}
      g_{\boldsymbol{y}}(\boldsymbol{x}) = \boldsymbol{y} + \boldsymbol{x} - f(\boldsymbol{x}) \quad \Rightarrow \quad \boldsymbol{y} - f(\boldsymbol{x}) = g_{\boldsymbol{y}} - \boldsymbol{x}.
    \end{equation*}
    Note that $Dg_{\boldsymbol{y}}(\boldsymbol{x}) = I - Df(\boldsymbol{x})$ and
    does not depend on $\boldsymbol{y}$. Also that
    $Dg_{\boldsymbol{y}}(\boldsymbol{0}) = I - I = 0$. 
    
    By continuity, we have $\|Dg_{\boldsymbol{y}}(\boldsymbol{x}) =
    \|Dg_{\boldsymbol{0}}(\boldsymbol{x})\| \leq 1/2$ for some $\boldsymbol{x}$
    near $\boldsymbol{0}$. This implies that
    \begin{equation*}
      \|g_{\boldsymbol{y}}(\boldsymbol{x}_1 - g_{\boldsymbol{y}}(\boldsymbol{x}_2)\| \leq \frac{1}{2}\|\boldsymbol{x}_1 - \boldsymbol{x}_2\|
    \end{equation*}
    for $\boldsymbol{x}_{1,2} \in D(0; r)$. Taking $\boldsymbol{x}_2 =
    \boldsymbol{0}$, we also get
    \begin{equation*}
      \|g_{\boldsymbol{y}}(\boldsymbol{x}) - \boldsymbol{y}\| \leq \frac{1}{2}\|\boldsymbol{x}\|,
    \end{equation*}
    so we have
    \begin{equation*}
      \|g_{\boldsymbol{y}}(\boldsymbol{x})\| \leq \frac{1}{2}\|\boldsymbol{x}\| + \|\boldsymbol{y}\|
    \end{equation*}
    for $\boldsymbol{y} \in D(0; r/2)$ and $\boldsymbol{x} \in D(0; r)$, and
    thus $\|g_{\boldsymbol{y}}(\boldsymbol{x})\| \leq r$. Hence we have
    $g_{\boldsymbol{y}}(x) : D(0; r) \to D(0; r)$, and $g_{\boldsymbol{y}}(x)$
    is by construction a contraction since
    $\|g_{\boldsymbol{y}}(\boldsymbol{x}_1 -
    g_{\boldsymbol{y}}(\boldsymbol{x}_2)\} \leq (1/2)\|\boldsymbol{x}_1 -
    \boldsymbol{x}_2\|$.
  
    \item By contraction mapping theorem, for all $\boldsymbol{y} \in D(0; r/2)$,
    there exists a unique $\boldsymbol{x} \in D(0; r)$ with $\boldsymbol{y} =
    f(\boldsymbol{x})$, so there exists an inverse function defined on $D(0;
    r/2)$.
    
    Define
    \begin{equation*}
      U_1 = \{\boldsymbol{x} \in U\ :\ \|\boldsymbol{x}\| < r, \|f(\boldsymbol{x})\| < r/2\}, \quad V_1 = f(U_1) = B(0; r/2).
    \end{equation*}
    By definition, both the domain and image are open sets. $f : U_1 \to V_1$ is a
    restricted bijection since it is a bijection on $D(0; r/2) \supset B(0; r/2)$.
    Given $\boldsymbol{x}_{1,2} \in D(0; r)$, we have
    \begin{align*}
      \|\boldsymbol{x}_1 - \boldsymbol{x}_2\| &= \| g_{\boldsymbol{0}}(\boldsymbol{x}_1) + f(\boldsymbol{x}_1) - g_{\boldsymbol{0}}(\boldsymbol{x}_2) + f(\boldsymbol{x}_2)\| \\
        & \leq \|g_{\boldsymbol{0}} - g_{\boldsymbol{0}}(\boldsymbol{x}_2)\| + \|f(\boldsymbol{x}_1) - f(\boldsymbol{x}_2)\| \\
        & \leq \frac{1}{2} \|\boldsymbol{x}_1 - \boldsymbol{x}_2\| + \|f(\boldsymbol{x}_1) - f(\boldsymbol{x}_2)\|,
    \end{align*}
    so that $\|\boldsymbol{x}_1 - \boldsymbol{x}_2\| \leq 2\|f(\boldsymbol{x}_1) -
    f(\boldsymbol{x}_2)\|$. For $\boldsymbol{x}_2 = 0$, we have
    $\|\boldsymbol{x}_1\| \leq 2\|f(\boldsymbol{x}_1)\|$. Since
    $\|f(\boldsymbol{x}_1)\| < r/2$ by construction, we have $\|\boldsymbol{x}_1\|
    < r$, so indeed $V_1 = B(0; r/2)$.
    
    For $f^{-1} = \phi$, $\|\boldsymbol{x}_1 - \boldsymbol{x}_2\| \leq
    2\|f(\boldsymbol{x}_1) - f(\boldsymbol{x}_2)\|$ implies that
    $\|\phi(\boldsymbol{y}_1) - \phi(\boldsymbol{y}_2)\| \leq 2\|\boldsymbol{x}_1
    - \boldsymbol{x}_2\|$, so that $f^{-1}$ is Lipschutz continuous.
    
    \item Note that $Df(\boldsymbol{x})$ is invertible for all $\boldsymbol{x}
    \in D(0; r)$, since we have $g_{\boldsymbol{0}}(\boldsymbol{x}) -
    \boldsymbol{x} = f(\boldsymbol{x})$, so that $Df(\boldsymbol{x}) = I -
    Dg_{\boldsymbol{0}}(\boldsymbol{x})$, but
    $\|Dg_{\boldsymbol{0}}(\boldsymbol{x})\| \leq 1/2$ from point 2 above, so
    $Df(\boldsymbol{x})$ is invertible for all $x \in D(0;r)$, and in particular
    for $x \in B(0; r) \subset D(0; r)$.
    
    \item Recall that if $f$ id differentiable, then $f(\boldsymbol{x}_1) -
    f(\boldsymbol{x}_2) = Df(\boldsymbol{x}_1)(\boldsymbol{x}_1 -
    \boldsymbol{x}_2) + R(\boldsymbol{x}_1 - \boldsymbol{x}_2)$ with
    $R(\boldsymbol{h}) / \|\boldsymbol{h}\| \to 0$ as $\|\boldsymbol{h}\| \to
    0$. Let $\boldsymbol{y}_i = f(\boldsymbol{x}_i)$. For $i=1,2$,
    \begin{equation*}
      \boldsymbol{y}_1 - \boldsymbol{y}_2 = Df(\boldsymbol{x}_1)\left(\phi(\boldsymbol{y}_1) - \phi(\boldsymbol{y}_2)\right) + R\left(\phi(\boldsymbol{y}_1) - \phi(\boldsymbol{y}_2)\right),
    \end{equation*}
    so that
    \begin{align*}
      \left(Df(\phi(\boldsymbol{y}_1))\right)^{-1}(\boldsymbol{y}_1 - \boldsymbol{y}_2) &= \left(\phi(\boldsymbol{y}_1) - \phi(\boldsymbol{y}_2)\right) \\
        &+\left(Df(\phi(\boldsymbol{y}_1))\right)^{-1}R\left(\phi(\boldsymbol{y}_1) - \phi(\boldsymbol{y}_2)\right).
    \end{align*}
    
    We want to show that the remainder term tends to zero, which will show that
    $\phi = f^{-1}$ is differentiable. For that, note we have, by
    Cauchy--Schwartz and point 3 above,
    \begin{align*}
      \frac{\|\left(Df(\phi(\boldsymbol{y}_1))\right)^{-1}R\left(\phi(\boldsymbol{y}_1) - \phi(\boldsymbol{y}_2)\right)\|}{\|\boldsymbol{y}_1 - \boldsymbol{y}_2\|} \leq  \frac{\|\left(Df(\phi(\boldsymbol{y}_1))\right)^{-1}\|\cdot \|R\left(\phi(\boldsymbol{y}_1) - \phi(\boldsymbol{y}_2)\right)\|}{(1/2)\|\left(\phi(\boldsymbol{y}_1) - \phi(\boldsymbol{y}_2)\right)\|}.
    \end{align*}
    $\left(Df(\phi(\boldsymbol{y}_1))\right)^{-1}$ is bounded since $f$ is
    differentiable. Further more, $f$ differentiable means
    $\|R\left(\phi(\boldsymbol{y}_1) - \phi(\boldsymbol{y}_2)\right)\| /
    \|\left(\phi(\boldsymbol{y}_1) - \phi(\boldsymbol{y}_2)\right)\| \to 0$ as
    $\|\left(\phi(\boldsymbol{y}_1) - \phi(\boldsymbol{y}_2)\right)\| \to 0$.
    Thus the desired remainder goes to zero since $\boldsymbol{y}_1 -
    \boldsymbol{y}_2 \to 0$ implies $\phi(\boldsymbol{y}_1) -
    \phi(\boldsymbol{y}_2) \to 0$, and $\phi = f^{-1}$ is differentiable.
    
    \item The derivative $D\phi(\boldsymbol{y}) =
    (Df(\phi(\boldsymbol{y})))^{-1} = \mbox{inv} \circ Df \circ
    \phi)\boldsymbol{y}$, so by construction, $D\phi = Df^{-1}$ is continuous.
    By chain rule, if $f\in C^k$, $D^{k-1}\phi$ is continuous, and thus $\phi =
    f^{-1} \in C^k$.
  \end{itemize}
  
   \qed
\end{proof}

%%%%%%%%%%%%%%%%%%%%%%%%%%%%%%

\subsection{Implicit function theorem}

%%%%%%%%%%%%%%%%%%%%%%%%%%%%%%

\subsection{Manifolds}

%===============================================================================

\chapter{Tangent spaces and vector fields}

%-------------------------------------------------------------------------------

\section{Method of Lagrange multipliers}

%===============================================================================

\chapter{Differential forms on $\mathbb{R}^n$}

%===============================================================================

\chapter{Differential forms on oriented manifolds}

%===============================================================================

%%%%%%%%%%%%%%%%%%%%%%%%%%%%%%%%%%%%%%%%%

% r.5 contents
%\tableofcontents

%\listoffigures

%\listoftables

% r.7 dedication
%\cleardoublepage
%~\vfill
%\begin{doublespace}
%\noindent\fontsize{18}{22}\selectfont\itshape
%\nohyphenation
%Dedicated to those who appreciate \LaTeX{} 
%and the work of \mbox{Edward R.~Tufte} 
%and \mbox{Donald E.~Knuth}.
%\end{doublespace}
%\vfill

% r.9 introduction
% \cleardoublepage

%%%%%%%%%%%%%%%%%%%%%%%%%%%%%%%%%%%%%%%%%
% actual useful crap (normal chapters)
\mainmatter

%\part{Basics (?)}


%\backmatter

%\bibliography{refs}
\bibliographystyle{plainnat}

%\printindex

\end{document}

